% Charles McEachern

% Fall 2015

% This is a template for writing a note to Bob in TeX. 

\documentclass{article}

% =============================================================================
% ==================================== Package List Copied from Thesis Template
% =============================================================================

\usepackage{epsfig} % Allows the inclusion of eps files
\usepackage{epic} % Enhanced picture mode
\usepackage{eepic} % Extensions for epic
\usepackage{units} % SI unit typesetting
\usepackage{url} % URL handling
\usepackage{longtable} % Tables that continue onto multiple pages
\usepackage{mathrsfs} % Support for \mathscr script
\usepackage{multirow} % Span rows in tables
\usepackage{bigstrut} % Space struts in tables up and down
\usepackage{amssymb} % AMS math symbols and helpers
\usepackage{graphicx} % Enhanced graphics support
\usepackage{setspace} % Adjust spacing in captions, single by default
\usepackage{xspace} % Automatically adjusting space after macros
\usepackage{amsmath} % \text, and other math formatting options
\usepackage{siunitx} % \num{} formatting and SI unit formatting
\usepackage{booktabs} % Enhanced tables with \toprule, etc.
\usepackage{hyperref} % Add clickable links to other parts of the document
\usepackage[noabbrev]{cleveref} % Automatically determine \cref type

% Configure the siunitx package
\sisetup{
    group-separator = {,}, % Use , to separate groups of digits, like 12,345
    list-final-separator = {, and } % Always use the serial comma in \SIlist
}

% Configure the cleveref package
\newcommand{\creflastconjunction}{, and } % Always use the serial comma

\linespread{1.3}

% =============================================================================
% ================================================================= Definitions
% =============================================================================

\newcommand{\Alfven}{Alfv\'en\xspace}

\newcommand{\Ampere}{Amp\`ere\xspace}

% =============================================================================
% ======================================================== Document Starts Here
% =============================================================================

\begin{document}




\section{Stuff to Copy-Paste for AGU Poster}



\begin{align*}
  \omega^2 & = k^2 v_A^2 \qquad
  \mathrm{so} \qquad 
  \omega_{min}^2 = k_\bot^2 v_A^2 \qquad
  \mathrm{or} \qquad
  \omega_{min} \sim \frac{m}{2 \pi r} v_A
\end{align*}


\begin{align*}
  \omega^2 & = k^2 v_A^2 \qquad
  \mathrm{so} \qquad 
  \omega_{min}^2 = k_\bot^2 v_A^2 \qquad
  \mathrm{or} \qquad
  f_{min} = \frac{m \, v_A}{r}
\end{align*}




\begin{align*}
  0 & = \sigma_\parallel E_\parallel - J_\parallel \;\;
  \rightarrow \;\;
  \frac{m \sigma_\parallel}{n e^2} \frac{\partial}{\partial t} J_\parallel = 
  \sigma_\parallel E_\parallel - J_\parallel \qquad
  \mathrm{with} \qquad
  \epsilon_0 \;\;
  \rightarrow \;\;
  \epsilon_\parallel
\end{align*}






Perpendicular Ampere’s law, after using Ohm’s law to eliminate current. 
\begin{align*}
  E_\parallel \! ( \delta t )& = 
    E_\parallel \! ( 0 ) \exp ( {\scriptstyle -\frac{\sigma_0 \delta t}{\epsilon_0} } ) \; +\;  
    \delta t \, 
    F_\parallel \! ( {\scriptstyle \frac{\delta t}{2} } ) 
    \exp ( {\scriptstyle -\frac{\sigma_0 \delta t}{2 \epsilon_0} } )
\end{align*}




Maxwell’s equations, all on one line (probably better for the poster)
\begin{align*}
  \frac{\partial}{\partial t} \underline{B} =
  - \nabla \times \underline{E} \qquad ; \qquad
  \underline{ \underline{\epsilon} } \cdot \frac{\partial}{\partial t} \underline{E} =
  \frac{1}{\mu_0} \nabla \times \underline{B} - 
  \underline{J}_{drive} - 
  \underline{J} \qquad ; \qquad
  0 =
  \underline{ \underline{\sigma} } \cdot \underline{E} - \underline{J}
\end{align*}



Perpendicular Ampere’s law, after using Ohm’s law to eliminate current. 
\begin{align*}
  \epsilon_\bot \frac{\partial}{\partial t} \underline{E}_\bot &=
  \underbrace{
  \left[ \frac{1}{\mu_0} \nabla \times \underline{B} - \underline{J}_{drive} \right]_\bot 
  }_{ \underline{F}_\bot } - 
  \underline{ \underline{\sigma} }_\bot \! \cdot \underline{E}_\bot \qquad ; \qquad
  \underline{ \underline{R} }( \theta ) & = 
  \left[
  \begin{array}{cc}
    \cos\theta & -\sin\theta \\
    \sin\theta & \cos\theta \\
  \end{array}
  \right]
\end{align*}



Perpendicular Ampere’s law, after using Ohm’s law to eliminate current. 
\begin{align*}
  \epsilon_\bot \frac{\partial}{\partial t} \underline{E}_\bot &=
  \underbrace{
  \left[ \frac{1}{\mu_0} \nabla \times \underline{B} - \underline{J}_{drive} \right]_\bot 
  }_{ \underline{F}_\bot } - 
  \underline{ \underline{\sigma} }_\bot \! \cdot \underline{E}_\bot
\end{align*}




Solution using integrating factors. 
\begin{align*}
  \underline{E}_\bot \! ( \delta t )& = \underline{ \underline{R} } 
    ( {\scriptstyle \frac{\sigma_H \delta t}{\epsilon_\bot} } ) \cdot 
    \underline{E}_\bot \! ( 0 ) \exp ( {\scriptstyle -\frac{\sigma_P \delta t}{\epsilon_\bot} } ) \; +\;  
    \delta t \, \underline{ \underline{R} } 
    ( {\scriptstyle \frac{\sigma_H \delta t}{2 \epsilon_\bot} } ) \cdot
    \underline{F}_\bot \! ( {\scriptstyle \frac{\delta t}{2} } ) 
    \exp ( {\scriptstyle -\frac{\sigma_P \delta t}{2 \epsilon_\bot} } )
\end{align*}



Definition of a rotation matrix
\begin{align*}
  \underline{ \underline{R} }( \theta ) & = 
  \left[
  \begin{array}{cc}
    \cos\theta & -\sin\theta \\
    \sin\theta & \cos\theta \\
  \end{array}
  \right]
\end{align*}



Explanation of breaking down the atmospheric magnetic field into harmonics
\begin{align*}
  \nabla \cdot \underline{B} = 0 \qquad 
  \mathrm{and} \qquad 
  \nabla \times \underline{B} = 0 \qquad 
  \mathrm{so} \qquad 
  \nabla^2 \Psi = 0 \qquad
  \mathrm{where} \qquad
  \nabla \Psi = \underline{B}
\end{align*}



Atmospheric jump condition
\begin{align*}
  \mu_0 \, \underline{ \underline{\Sigma} } \cdot \underline{E} & = 
  \hat{r} \times \Delta \underline{B}
\end{align*}
This is present in Lysak 2013. Integrating Ampere’s law over the atmosphere? 




Probably don't need anything past here... \\ 



Maxwell’s equations, one per line. 
\begin{align*}
  \frac{\partial}{\partial t} \underline{B} &=
  - \nabla \times \underline{E} \\
  \underline{ \underline{\epsilon} } \cdot \frac{\partial}{\partial t} \underline{E} &=
  \frac{1}{\mu_0} \nabla \times \underline{B} - 
  \underline{J}_{drive} - 
  \underline{J} \\
  0 & =
  \underline{ \underline{\sigma} } \cdot \underline{E} - \underline{J}
\end{align*}



\begin{align*}
  \frac{\partial}{\partial t} \underline{B} &=
    -\nabla \times \underline{E} & 
  \underline{ \underline{\epsilon} } \cdot \frac{\partial}{\partial t} \underline{E} &=
    \frac{1}{\mu_0} \nabla \times \underline{B} - \underline{J}_D - \underline{J}_O &
  0 &=
    \underline{ \underline{\sigma} } \cdot \underline{E} - \underline{J}_O
\end{align*}




\begin{align*}
  \underline{ \underline{\epsilon} } \cdot \frac{\partial}{\partial t} \underline{E} &=
    \left[ \frac{1}{\mu_0} \nabla \times \underline{B} - 
    \underline{J}_D \right] - 
    \underline{ \underline{\sigma} } \cdot \underline{E}
\end{align*}



Junk follows...



\begin{align*}
  \epsilon_\bot \frac{\partial}{\partial t} \underline{E}_\bot &=
    \left( \frac{1}{\mu_0} \nabla \times \underline{B} - 
    \underline{J}_D \right)_\bot - 
    \underline{ \underline{\sigma} }_\bot \cdot \underline{E}_\bot
\end{align*}






\begin{align*}
  \epsilon_\bot \frac{\partial}{\partial t} \underline{E}_\bot &=
    \underbrace{ \left( \frac{1}{\mu_0} \nabla \times \underline{B} - 
    \underline{J}_D \right)_\bot }_{ \underline{F}_\bot } - 
    \underline{ \underline{\sigma} }_\bot \cdot \underline{E}_\bot
\end{align*}




\begin{align*}
  \underline{E}_\bot \! \left( \delta t \right) &= 
    \underline{ \underline{R} } \left( \frac{ \sigma_H \, \delta t }{\epsilon_\bot} \right) \cdot
    \underline{E}_\bot \! \left( 0 \right) \;
      \exp \left( - \frac{ \sigma_P \, \delta t }{\epsilon_\bot} \right) + 
    \underline{ \underline{R} } \left( \frac{\sigma_H \, \delta t}{2 \epsilon_\bot} \right) \cdot
    \delta t \, \underline{F}_\bot \! \left( \frac{\delta t}{2} \right) \; \exp \left( - \frac{ \sigma_P \, \delta t }{ 2 \epsilon_\bot} \right)
\end{align*}






\begin{align*}
  \underline{ \underline{R} } \left( \theta \right) & = 
    \left( \begin{array}{cc}
    \cos\theta & -\sin\theta \\
    \sin\theta & \cos\theta \end{array} \right)
\end{align*}








\begin{align*}
  \frac{\partial}{\partial t} \underline{B} &=
    -\nabla \times \underline{E} & 
  \underline{ \underline{\epsilon} } \cdot \frac{\partial}{\partial t} \underline{E} &=
    \frac{1}{\mu_0} \nabla \times \underline{B} - \underline{J} &
  0 &=
    \underline{ \underline{\sigma} } \cdot \underline{E} - \underline{J}
\end{align*}






\begin{align*}
  \underline{ \underline{\epsilon} } \cdot \frac{\partial}{\partial t} \underline{E} &=
    \frac{1}{\mu_0} \nabla \times \underline{B} - 
    \underline{J}_D - 
    \underline{J}_O
\end{align*}


\begin{align*}
  \underline{ \underline{\epsilon} } \cdot \frac{\partial}{\partial t} \underline{E} &=
    \frac{1}{\mu_0} \nabla \times \underline{B} - 
    \underline{J}_{D} - 
    \underline{ \underline{\sigma} } \cdot \underline{E}
\end{align*}


\begin{align*}
  \underline{ \underline{\epsilon} } \cdot \frac{\partial}{\partial t} \underline{E} &=
    \underbrace{ \frac{1}{\mu_0} \nabla \times \underline{B} - 
    \underline{J}_{Drive} }_{ \underline{F} } - 
    \underline{ \underline{\sigma} } \cdot \underline{E}
\end{align*}





\begin{align*}
  f = \underbrace{x^3}_\textrm{text 1} + \underbrace{2}_\textrm{text 2}
\end{align*}














\section{Dispersion Relation Setup}

Start with Faraday's Law, \Ampere's Law, and Ohm's Law. Break the latter two
into parallel and perpendicular components. 
\begin{align}
  \frac{\partial}{\partial t} \underline{B} &= -\nabla \times \underline{E} \\
  \epsilon_\bot \frac{\partial}{\partial t} \underline{E}_\bot &= 
    \frac{1}{\mu_0} \left( \nabla \times \underline{B} \right)_\bot -
    \underline{J}_\bot \\
  \epsilon_\parallel \frac{\partial}{\partial t} E_\parallel & = 
    \frac{1}{\mu_0} \left( \nabla \times \underline{B} \right)_\parallel -
    J_\parallel \\
  \label{ohm_perp}
  0 & = \underline{\underline{\sigma}}_\bot \cdot \underline{E}_\bot - 
    \underline{J}_\bot \\
  \label{ohm_par}
  \frac{m \sigma_0}{n e^2} \frac{\partial}{\partial t} J_\parallel & = 
    \sigma_0 E_\parallel - J_\parallel
\end{align}

Note that we include the electron inertial effects in \cref{ohm_par}, the 
parallel component of Ohm's Law, but not in \cref{ohm_perp}, the perpendicular 
components. This is justified by the fact that the parallel conductivity is 
much larger than the perpendicular conductivity. Velocity- and 
temperature-dependent terms are neglected completely; we are concerned with 
waves much faster than convective timescales. 

We're looking for plane wave solutions: fields which (at least locally) evolve 
with $\exp ( i \underline{k} \cdot \underline{x} - i \omega t)$. That lets us 
evaluate the spatial and temporal derivatives in the above equations. 

First, we use Ohm's Law to eliminate $\underline{J}$. 
\begin{align}
  \label{new_faraday}
  - i \omega \underline{B} &= - i \underline{k} \times \underline{E} \\
  \label{new_perp}
  - i \omega \epsilon_\bot \underline{E}_\bot &= 
    \frac{i}{\mu_0} \left( \underline{k} \times \underline{B} \right)_\bot -
    \underline{\underline{\sigma}}_\bot \cdot \underline{E}_\bot \\
  \label{new_par}
  - i \omega \epsilon_\parallel E_\parallel & = 
    \frac{i}{\mu_0} \left( \underline{k} \times \underline{B} 
      \right)_\parallel -
    \frac{\sigma_0}{1 - \frac{i \omega m \sigma_0}{n e^2}} E_\parallel
\end{align}

To eliminate $\underline{B}$, we take the curl of \cref{new_faraday} and 
substitute into \cref{new_perp,new_par}. Note that 
$\underline{k} \times \underline{k} \times \underline{E}$ 
can be written as 
$\underline{k} \big( \underline{k} \cdot \underline{E} \big) - k^2 
\underline{E}$, 
and 
$\underline{k} \cdot \underline{E} = 0$ 
in the case of quasi-neutrality. 
\begin{align}
  - i \omega \epsilon_\bot \underline{E}_\bot &= 
    \frac{i k_\bot^2}{\mu_0 \omega} 
      \underline{E}_\bot -
    \underline{\underline{\sigma}}_\bot \cdot \underline{E}_\bot \\
  - i \omega \epsilon_\parallel E_\parallel & = 
    \frac{i k_\parallel^2}{\mu_0 \omega} E_\parallel -
    \frac{\sigma_0}{1 - \frac{i \omega m \sigma_0}{n e^2}} E_\parallel
\end{align}

Taking $c_\parallel^2 \equiv \frac{1}{\mu_0 \epsilon_\parallel}$ (where 
$\epsilon_\parallel$ is maybe not the same as $\epsilon_0$), and using 
$\sigma_\parallel$ to represent 
$\frac{\sigma_0}{1 - \frac{i \omega m \sigma_0}{n e^2}}$, 
we rearrange these to:
\begin{align}
  \label{perp_final}
  \left[ \frac{k_\bot c^2}{\omega^2} \underline{\underline{\mathbb{I}}} - 
    \frac{\epsilon_\bot}{\epsilon_0} \underline{\underline{\mathbb{I}}} - 
    \frac{i}{\epsilon_0 \omega} \underline{\underline{\sigma}}_\bot \right] 
    \cdot \underline{E}_\bot &= 0 \\
  \label{par_final}
  \left[ \frac{k_\parallel c_\parallel^2}{\omega^2} - 
    1 - 
    \frac{i}{\epsilon_\parallel \omega} \sigma_\parallel \right] 
    E_\parallel &= 0
\end{align}

To obtain a dispersion relation, we combine \cref{perp_final,par_final} into 
an equation of the form $\underline{\underline{D}} \cdot \underline{E} = 0$, 
then solve for $| \underline{\underline{D}} | = 0$. Conveniently, with the 
parallel and perpendicular components decoupled, we can take shortcut: 
\begin{align}
  \left( \frac{k_\parallel c_\parallel^2}{\omega^2} - 1 - 
    \frac{i}{\epsilon_\parallel \omega} \sigma_\parallel \right)
    \left| \frac{k_\bot c^2}{\omega^2} \mathbb{I} - 
    \frac{\epsilon_\bot}{\epsilon_0} \mathbb{I} - 
    \frac{i}{\epsilon_0 \omega} \underline{\underline{\sigma}}_\bot \right| 
      &= 0
\end{align}

\section{Easy Roots}

Before getting into the perpendicular tensor, let's take a look at the easy 
roots. These (supposing they exist and are physical) will suggest wave modes 
with nonzero parallel electric field components. 
\begin{align}
  \label{easy_roots}
  \frac{k_\parallel c_\parallel^2}{\omega^2} - 1 - 
    \frac{i}{\epsilon_\parallel \omega} \sigma_\parallel &= 0
\end{align}

Due to the $\omega$ dependence in $\sigma_\parallel$, this is actually quartic 
in $\omega$ (and sadly \emph{not} quadratic in $\omega^2$). However, it's 
quadratic in two regimes, both of which are of interest. 

\subsection{Small $\sigma_0$ Limit}

Near the ionospheric boundary of the simulation, for $\omega$ in the ULF 
range, we find $\frac{\omega m \sigma_0}{n e^2}$ as small as $10^{-5}$, so 
$\sigma_\parallel \approx \sigma_0$. That is, in this regime, the electron 
inertial term is not significant. 

In this limit, solutions to \cref{easy_roots} are given by:
\begin{align}
  \omega &= - \frac{i \sigma_0}{2 \epsilon_\parallel} \left( 1 \pm 
    \sqrt{ 1 - 
      \frac{4 c_\parallel^2 k_\parallel^2 \epsilon_\parallel^2}{\sigma_0^2}
    } \right)
\end{align}

In the limit where 
$\frac{c_\parallel k_\parallel \epsilon_\parallel}{\sigma_0}$ is very large, 
this becomes $\omega = c_\parallel k_\parallel$. However, we are not at that 
limit; we see $\frac{c \epsilon_0}{\sigma_0} \sim \SI{1}{\meter}$. As a 
result, the mode described by this dispersion relation is negative and 
imaginary (that is, evanescent). 

\subsection{Large $\sigma_0$ Limit}

Far from the ionosphere, $\sigma_0$ becomes very large, so 
$\sigma_\parallel \approx \frac{i n e^2}{\omega m}$. Plugging that in, 
\cref{easy_roots} simplifies to:
\begin{align}
  \omega^2 &= c_\parallel^2 k_\parallel^2 + \omega_{p \parallel}^2
\end{align}

This looks like the plain old electromagnetic wave, but it's polarized (at 
least in part) along the magnetic field. That makes it... the O-mode?

\section{Hard Roots}

To be continued... 

\end{document}













