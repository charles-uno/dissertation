% %%%%%%%%%%%%%%%%%%%%%%%%%%%%%%%%%%%%%%%%%%%%%%%%%%%%%%%%%%%%%%%%%%%%%%%%%%%%%
% %%%%%%%%%%%%%%%%%%%%%%%%%%%%%%%%%%%%%%%%%%%%%%%%%%%%%%%%%%%%%%%%%%%%%%%%%%%%%
% %%%%%%%%%%%%%%%%%%%%%%%%%%%%%%%%%%%%%%%%%%%%%%%%%%%%%%%%%%%%%%%%%%%%%%%%%%%%%

\chapter{Introduction}
  \label{ch_intro}

\todo{In 1859, humanity was working hard to get its shit together.} The United States moved steadily toward the American Civil War, which would abolish slavery and consolidate the power of the federal government. A slew of conflicts in Southern Europe, such as the Austro-Sardinian War, set the stage for the unification of Italy. The Taiping Civil War -- one of the bloodiest conflicts of all time -- is considered by many to mark the beginning of modern Chinese history. Origin of Species was published. The first transatlantic telegraph cable was laid.

Ambivalent to humanity, the Sun belched out an intense burst of charged particles and magnetic energy\footnote{Such bursts are called coronal mass ejections (CMEs).}, aimed directly at Earth. The resulting geomagnetic storm\footnote{The Solar Storm of 1859 is also called the Carrington Event, after English astronomer Richard Carrington, who drew a connection between the storm's geomagnetic activity and the sunspots he had observed the day before.} caused telegraph systems to fail across the Western hemisphere\cite{green_2006}, electrocuting some operators. Displays of the northern lights were visible as far south as Cuba. 

A coronal mass ejection of similar magnitude narrowly missed Earth in 2012\cite{nasa_2012}. Had it not, it's estimated\cite{lloyds_2013} that it would have caused widespread, long-term electrical outages, with a damage toll on the order of \num{e9} dollars. 

The Sun's extreme -- and temperamental -- effect on Earth's magnetic environment makes a compelling case for the ongoing study of space weather. Such research has evolved over the past century from sunspot counts and compass readings to multi-satellite missions and supercomputer simulations. Modern methods have dramatically increased humanity's understanding of the relationship between the Sun and the Earth; however, significant uncertainty continues to surround geomagnetic storms, substorms, and the various energy transport mechanisms that make them up. 

\todo{Giant pulsations preferrentially occur during quiet times... do we even want to bring up storms here? }

The present work focuses in particular on the phenomenon of field line resonance: \Alfven waves guided by Earth's magnetic field lines, bouncing between the northern and southern ionospheres. \todo{Such waves are exciting because they energize particles... and move energy down to the ionosphere? }

\nopagebreak





%\todo{The storm of 1859 presented compelling evidence that the Sun drives geomagnetic activity. In the decades that followed, a model took shape to describe the mechanisms of energy transfer between the Sun and the Earth. Based on auroral observations and data from ground-based magnetometers, Birkeland argued for the existence of a constant outflow of electrons and ions from the Sun -- the solar wind. The advent of high-frequency radio communication allowed Kennelly, Heaviside, and others to probe the electrical properties of the upper atmosphere. }

%\todo{The study of space weather was revolutionized by the development of sounding rockets and satellites in the mid twentieth century. This allowed direct observation of the structure of the near-Earth environment, including, crucially, the waves that carry energy through it. }

%\todo{Not least among these advances was the discovery of \Alfven waves. {\Alfven}ic aurora. Carry energy and particles. }

%The study of space weather revolves around the transfer of energy from the Sun to the Earth. Ultra low frequency waves in particular are an important energy transport mechanism between the magnetosphere's outer boundary (at the solar wind) and its inner boundary (at the top of the atmosphere). 







