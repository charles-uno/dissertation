% %%%%%%%%%%%%%%%%%%%%%%%%%%%%%%%%%%%%%%%%%%%%%%%%%%%%%%%%%%%%%%%%%%%%%%%%%%%%%
% %%%%%%%%%%%%%%%%%%%%%%%%%%%%%%%%%%%%%%%%%%%%%% Description of Numerical Model
% %%%%%%%%%%%%%%%%%%%%%%%%%%%%%%%%%%%%%%%%%%%%%%%%%%%%%%%%%%%%%%%%%%%%%%%%%%%%%

\chapter{Model}
  \label{ch_model}


\todo{Numerical simulations, by their nature, have to make assumptions about the nature of the magnetosphere. In exchange, they can provide a whole-magnetosphere picture of phenomena like field line resonance. }








\todo{The present chapter sketches the fundamental equations of waves in a cold, resistive plasma. Then illustrates the implementation of a two and a half dimensional \Alfven wave code to model those waves. Finally, takes a look at a little something extra that the code might like. }


%``Increasing the Hall conductance allows the energy to oscillate through the inductive process rather than dissipate as Joule heating, increasing the `ringtime' of field line resonances.''\cite{waters_2013}

Ionospheric profiles are static for the duration of a simulation. Even so-called ultra low frequency waves are still much faster than convective timescales. Each profile is resolved to an altitude of about $\SI{e4}{\km}$, and include well-resolved $E$, $F_1$, and $F_2$ layers. 

