% %%%%%%%%%%%%%%%%%%%%%%%%%%%%%%%%%%%%%%%%%%%%%%%%%%%%%%%%%%%%%%%%%%%%%%%%%%%%%
% %%%%%%%%%%%%%%%%%%%%%%%%%%%%%%%%%%%%%%%%%%%%%% Description of Numerical Model
% %%%%%%%%%%%%%%%%%%%%%%%%%%%%%%%%%%%%%%%%%%%%%%%%%%%%%%%%%%%%%%%%%%%%%%%%%%%%%

\chapter{Model}
  \label{ch_model}

\todo{This chapter runs through the numerical model... starting with the math that underlies it, then looking at the model itself, then a cool extra thing. }







``Increasing the Hall conductance allows the energy to oscillate through the inductive process rather than dissipate as Joule heating, increasing the `ringtime' of field line resonances.''\cite{waters_2013}



Ionospheric profiles are static for the duration of a simulation. Even so-called ultra low frequency waves are still much faster than convective timescales. Each profile is resolved to an altitude of about $\SI{e4}{\km}$, and include well-resolved $E$, $F_1$, and $F_2$ layers. 

