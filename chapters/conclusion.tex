


\chapter{Conclusion}
  \label{ch_conclusion}


% -----------------------------------------------------------------------------
% -----------------------------------------------------------------------------
% -----------------------------------------------------------------------------
\section{Summary of Results}

\todo{Code development. }

\todo{Numerical results. }

\todo{Observational results. }

% -----------------------------------------------------------------------------
% -----------------------------------------------------------------------------
% -----------------------------------------------------------------------------
\section{Future Work}

\todo{Code development. }

Arbitrary deformation of grid. Get $\hat{e}_i = \frac{\partial}{\partial \lysaki} \underline{x}$, then $g_{ij} = \hat{e}_i \cdot \hat{e}_j$, then invert the metric tensor for contravariant components.  

MPI. Some benchmarks with time to compute vs time to broadcast. This might make sense for inertial length scales. 

Better ionospheric profiles. Distinction between the dawn and dusk flanks. Maybe even update the conductivity based on energy deposition --- precipitation causes ionization! 

IRI ionosphere model. Solar illumination effects. 

\todo{Numerical work. }

More complicated driving. Higher harmonics, non-sinusoidal waveforms. Maybe evern drive based on events/Dst? 

Look at runs with a plasmasphere further out. Can we get nicer resonance on the nightside? A larger plasmapause will cause L=5 to line up better with Pc4 frequencies. 

\todo{Data analysis. }

Look at distribution of these events with respect to the plasmapause, storm phase... 




