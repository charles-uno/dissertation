


\chapter{Conclusion}
  \label{ch_conclusion}


% -----------------------------------------------------------------------------
% -----------------------------------------------------------------------------
% -----------------------------------------------------------------------------
\section{Summary of Results}

\todo{Code development... \cref{ch_model,ch_inertia} }

\todo{Make the Git repository public, and link to it. }

\todo{Numerical results... \cref{ch_results} }

\todo{Re-summarize the Discussion sections, I guess. }

\todo{Observational results... \cref{ch_rbsp} }

\todo{Link to the Git repository. }

% -----------------------------------------------------------------------------
% -----------------------------------------------------------------------------
% -----------------------------------------------------------------------------
\section{Future Work}

\todo{Code development. }

Arbitrary deformation of grid. Get $\hat{e}_i = \frac{\partial}{\partial \lysaki} \underline{x}$, then $g_{ij} = \hat{e}_i \cdot \hat{e}_j$, then invert the metric tensor for contravariant components.  

MPI. Time to compute vs time to broadcast. This might make sense for inertial length scales. 

Better ionospheric profiles. Distinction between the dawn and dusk flanks. Maybe even update the conductivity based on energy deposition --- precipitation causes ionization! 

IRI ionosphere model. Solar illumination effects. 

\todo{Numerical work. }

More complicated driving. Higher harmonics, non-sinusoidal waveforms. Maybe evern drive based on events? 

\todo{Analysis of RBSP data. }

Basically just do everything over again, twice as well, once the probes have finished sampling the dayside again. 






