


\chapter{Conclusion}
  \label{ch_conclusion}

\todo{$\cdots$}

% -----------------------------------------------------------------------------
% -----------------------------------------------------------------------------
% -----------------------------------------------------------------------------
\section{Code Development}

%\todo{Code development... \cref{ch_model,ch_inertia} }

%\todo{Make the Git repository public, and link to it. }

%Arbitrary deformation of grid. Get $\hat{e}_i = \frac{\partial}{\partial \lysaki} \underline{x}$, then $g_{ij} = \hat{e}_i \cdot \hat{e}_j$, then invert the metric tensor for contravariant components.  

%MPI. Time to compute vs time to broadcast. This might make sense for inertial length scales. 

%Better ionospheric profiles. Distinction between the dawn and dusk flanks. Maybe even update the conductivity based on energy deposition --- precipitation causes ionization! 

%IRI ionosphere model. Solar illumination effects. 

% -----------------------------------------------------------------------------
% -----------------------------------------------------------------------------
% -----------------------------------------------------------------------------
\section{Numerical Work}

%\todo{Numerical results... \cref{ch_results} }

%\todo{Re-summarize the Discussion sections, I guess. }

%More complicated driving. Higher harmonics, non-sinusoidal waveforms. Maybe evern drive based on events? 

%Look at the phase of waves in Tuna. How much is standing/traveling? 

% -----------------------------------------------------------------------------
% -----------------------------------------------------------------------------
% -----------------------------------------------------------------------------
\section{Van Allen Probe Pc4 Survey}

%\todo{Observational results... \cref{ch_rbsp} }

%\todo{Link to the Git repository. }

%Basically just do everything over again, twice as well, once the probes have finished sampling the dayside again. 






