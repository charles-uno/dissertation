


\chapter{Conclusion}
  \label{ch_conclusion}

Field line resonances in the Pc4 frequency range are known to exhibit a number
of heretofore-unconnected properties. The present work discusses the
development of Tuna, a numerical model created with Pc4s in mind. Using Tuna,
Pc4s are modeled across varying frequencies, azimuthal
modenumbers, and ionospheric conductivity profiles, suggesting novel
connections between Pc4 properties. Numerical results are complemented by a
survey of Pc4 events measured by the Van Allen Probes. 

% -----------------------------------------------------------------------------
% -----------------------------------------------------------------------------
% -----------------------------------------------------------------------------
\section{Code Development}

Tuna is a two and a half dimensional model --- it uses \maxeqs to simulate the
evolution of three-dimensional \Alfven waves on a two-dimensional slice of the
magnetosphere. Features include a nonorthogonal dipole geometry, a choice
between several height-resolved ionospheric conductivity tensors, and a novel
driving mechanism: ring current perturbation. Tuna has the capacity to simulate
parallel currents and electric fields through the consideration of electron
inertial effects, though those effects are disabled by default for the sake of
stability. Tuna's source code and support scripts are publicly available at
\url{https://github.com/UMN-Space-Physics}. 

Future development on Tuna could further parallelize the code using MPI; doing
so could allow the electron inertial length to be properly resolved (which
would in turn stabilize the electron inertial effects). Other potential
improvements include dynamic ionospheric profiles, which become more
conductive as energy is deposited in the ionosphere, deformation of the ideal
dipole grid, and north-south asymmetry in the ionospheric profile to account
for Earth's tilt. 

%\todo{Code development... \cref{ch_model,ch_inertia} }

%\todo{Make the Git repository public, and link to it. }

%Arbitrary deformation of grid. Get $\hat{e}_i = \frac{\partial}{\partial \lysaki} \underline{x}$, then $g_{ij} = \hat{e}_i \cdot \hat{e}_j$, then invert the metric tensor for contravariant components.  

%MPI. Time to compute vs time to broadcast. This might make sense for inertial length scales. 

%Better ionospheric profiles. Distinction between the dawn and dusk flanks. Maybe even update the conductivity based on energy deposition --- precipitation causes ionization! 

%IRI ionosphere model. Solar illumination effects. 

% -----------------------------------------------------------------------------
% -----------------------------------------------------------------------------
% -----------------------------------------------------------------------------
\section{Numerical Work}

Tuna suggests several novel results. 

Poloidal FLRs are shown to rotate to the toroidal mode on timescales comparable
to the oscillation period, suggesting poloidal waves as a significant source
for toroidal waves. On the nightside, the dissipation timescale is comparable
to the oscillation period as well, so much poloidal energy is lost before
rotating to the toroidal mode. 

Numerical results also suggest that the distinctive ground signatures
attributed to giant pulsations may be features of odd poloidal Pc4s overall.
Generic odd poloidal waves are shown to exhibit peak ground signatures at
${\azm = 16}$ to $32$, which are sharply peaked at auroral latitudes, and
which occur preferrentially during times of quiet solar activity. 

At low \azm, poloidal waves move readily across $L$-shells, allowing energy to
easily escape the simulation --- such waves can hardly even be called field
line resonances. At high \azm, the poloidal mode becomes guided, though it is
never defined as sharply in $L$ as the toroidal mode. This is consistent with
observations showing toroidal frequencies to depend strongly on $L$ compared to
poloidal frequencies. 

The present work exclusively used an odd, sinusoidal, poloidal driver; however,
Tuna has the capacity to deliver more interesting driving waveforms as well,
and higher harmonics could be added with trivial modifications to the code.
Test particles could be added, which would be particularly interesting in runs
including electron inertial effects (and thus parallel electric fields). Tuna
could also be used to consider ``in situ'' measurements of wave phase. 

%\todo{Numerical results... \cref{ch_results} }

%\todo{Re-summarize the Discussion sections, I guess. }

%More complicated driving. Higher harmonics, non-sinusoidal waveforms. Maybe evern drive based on events? 

%Look at the phase of waves in Tuna. How much is standing/traveling? 

% -----------------------------------------------------------------------------
% -----------------------------------------------------------------------------
% -----------------------------------------------------------------------------
\section{Van Allen Probe Pc4 Survey}

Using data from the Van Allen Probes EMFISIS and EMF instruments, the present
work shows \about500 half-hour Pc4 events, each classified by both polarization
and harmonic. Scripts used to download, process, and plot the data are
available at \url{https://github.com/UMN-Space-Physics}. 

Odd poloidal events are shown to be concentrated from midnight through the
morning --- as with the numerical results, it seems that giant pulsations are
unusual compared to Pc4s overall, but not compared to odd poloidal Pc4s
specifically. 

Odd toroidal events exhibit a similar distribution to odd poloidal events, but
are skewed dayward across the morningside. The same is true for even events:
even poloidal events peak at noon and are spread across the evening side, while
even toroidal events peak at noon and are far less spread. These distributions
are consistent with the poloidal mode as a significant source for same-harmonic
toroidal events, particularly on the dayside, as suggested by the numerical
results. 

Overall, poloidal and toroidal events exhibit disparate distributions because
poloidal events are primarily even, while toroidal events are mostly odd; the
cause is not apparent. 

The body of Pc4 events available in Van Allen Probe data is growing over time.
After the probes complete their second precession, event statistics on the
dayside should improve considerably, allowing more meaningful consideration of
subsets of the events, such as those that take place during geomagnetically
active times. Furthermore, the present work considered the two Van Allen Probes
to be independent observers, but future work could look into the few Pc4 events
which are observed simultaneously (or in quick succession) by both probes. This
could offer significant insight into the spatial structure and temporal
evolution of Pc4s --- particularly when complemented by numerical work! 


%\todo{Observational results... \cref{ch_rbsp} }

%\todo{Link to the Git repository. }

%Basically just do everything over again, twice as well, once the probes have finished sampling the dayside again. 






