


\chapter{The Near-Earth Environment}
  \label{ch_environment}


From Earth's surface to about \SI{100}{\km}, the atmosphere is a well-behaved fluid: a collisional ensemble of neutral atoms. However, beyond there, its behavior changes dramatically. As altitude increases, solar ultraviolet radiation becomes more intense, which ionizes atmospheric atoms. Density also decreases, slowing collisional recombination. Whereas the neutral atmosphere is held against Earth's surface by gravity, the behavior of charged particles is dominated by Earth's geomagnetic field... and the electromagnetic disturbances created as that field is hammered by the solar wind. 

\todo{Neutral density is larger than charged particle density (?) but it doesn't matter because mean free path is huge. }

The present section outlines the structure of the magnetosphere; that is, the region of space governed primarily by Earth's magnetic field. Particular emphasis is placed on structures which relate closely to field line resonance. 

% -----------------------------------------------------------------------------
% -----------------------------------------------------------------------------
% -----------------------------------------------------------------------------
\section{The Outer Magnetosphere}

During quiet times\footnote{Geomagnetically active times are the topic of \cref{ch_storms}}, the behavior of the magnetosphere is driven by the solar wind: a hot, rarefied, fast-moving solar outflow threaded by the Sun's magnetic field\footnote{Usual solar wind velocities, densities, and temperatures fall on the order of \SI{100}{\km/\s}, \SI{10}{\percc}, and \SI{1}{\keV} respectively. At Earth's orbit, the Sun's magnetic field has a magnitude around \SI{1}{\nT}. }. 

The magnetosphere's outer boundary represents a balance between the solar wind dynamic pressure and the magnetic pressure of Earth's dipole field. On the dayside, the dipole is compressed, pushing the magnetopause to within \SI{10}{\RE}\footnote{Distances in the magnetosphere are typically measured in units of Earth radii: $\SI{1}{\RE} \equiv \SI{6378}{\km}$. } of Earth. The nightside magnetosphere is stretched into a long tail which may exceed \SI{50}{\RE} in width and \SI{100}{\RE} in length. 

Consistent with \amplaw, the interplanetary magnetic field is separated from the magnetosphere by a current sheet: the magnetopause. On the dayside, the magnetopause current flows duskward; on the nightside, it flows dawnward around the magnetotail. 

Plasma within the tail is cool (\about\SI{100}{\eV}) and rarefied (\about\SI{e-2}{\percc}). Earth's dipole is significantly deformed in the magnetotail; the magnetic field in the northern lobe of the tail points more or less Earthward, and vice versa. The two lobes are divided by the plasma sheet, which is comparably hot (\about\SI{1}{\keV}) and dense (\about\SI{1}{\percc}). The plasma sheet carries a duskward current which connects to the magnetopause current. 

\todo{Reconnection. Dungey cycle. Slow accumulation of energy in the tail. }
\footnote{Closed field lines, such as those in the plasma sheet, connect at both ends to the magnetic dynamo at Earth's core. Open field lines, such as those in the tail lobes, meet Earth at only one end; the other connects to the interplanetary magnetic field. }
\footnote{In the outer magnetosphere (as well as most of the inner magnetosphere), collisions are so infrequent that magnetic flux is said to be ``frozen in'' to the plasma. Charged particles move freely along magnetic field lines, but cannot cross from one line to another. Compression of the magnetic field is synonymous with compression of the ambient plasma. }

% -----------------------------------------------------------------------------
% -----------------------------------------------------------------------------
% -----------------------------------------------------------------------------
\section{The Inner Magnetosphere}

Within about \todo{ $L\sim10$ }\footnote{The McIlwain parameter $L$ is a spatial parameter used to index field lines in Earth's dipole geometry: $L \equiv \frac{r}{\sin^2\theta}$ for colatitude $\theta$ and radius $r$ in Earth radii. For example, the $L=5$ field line passes through the equatorial plane at a geocentric radius of \SI{5}{\RE}, then meets the Earth at a colatitude of $\arcsin \sqrt{ \frac{1}{5} } \sim \SI{27}{\degree}$ (equally, a latitude of $\arcsin \sqrt{ \frac{1}{5} } \sim \SI{63}{\degree}$). }, the dipole magnetic field is not appreciably deformed by the solar wind. As a result, the structures in the inner magnetosphere follow closely from the motion of charged particles in an ideal dipole field. 

The plasmasphere -- a cold (\about\SI{1}{\eV}), dense (\SIrange{e2}{e4}{\percc}) torus of corotating plasma -- is formed by the outward drift of atmospheric ions along magnetic closed field lines. Its outer boundary, is thought to represent a balance between the corotation electric field (per the rotation of Earth's magnetic dipole) and the convection electric field (associated with the convection of magnetic flux during the Dungey cycle). Particle density drops sharply at the edge of the plasmasphere; the boundary is called the plasmapause. The plasmapause typically falls around $L=4$, though it can be significantly larger during prolonged quiet times. 

Between $L\sim3$ and $L\sim5$ lies a body of energetic (\SIrange{e3}{e5}{\eV}) electrons and ions undergoing gradient-curvature drift\footnote{Fundamental particle motions are discussed in \cref{ch_flrs}. }. Gradient-curvature drift carries positive and negative particles in opposite directions; the net result is a westward current called the ring current. Ring current intensity varies greatly based on geomagnetic conditions. During quiet times, it causes magnetic a southward magnetic field on the order of \SI{1}{\nT} at Earth's equator\footnote{For comparison, Earth's dipole field points north at the equator with a magnitude over \SI{e4}{\nT}. }. 

The inner magnetosphere is also home to the Van Allen radiation belts. The inner belt ($L\lesssim2$) and outer belt ($L\gtrsim4$) are primarily made up of high-energy (\SIrange{e5}{e7}{\eV}) protons and electrons respectively. 

\todo{Probably need a bit more about the radiation belts? Radial diffusion is a big deal. }

% -----------------------------------------------------------------------------
% -----------------------------------------------------------------------------
% -----------------------------------------------------------------------------
\section{The Ionosphere}
  \label{sec_ionos}

\todo{Structure of the ionosphere. Layers. Stratification of different species. UV ionization, charged particle precipitation, cosmic rays. }

\todo{Pedersen and Hall conductivity arise from a sweet spot where there are few enough collisions that ionization can be maintained, but enough to break adiabatic invariants and allow motion across field lines. }

\todo{Currents in the ionosphere create magnetic fields on the ground. }

\todo{Perpendicular currents close field-aligned currents connecting to the magnetopause or partial ring current. }

\todo{Two-cell convection? }

%From \cite{paschmann_2003}: ``In the thermosphere, the solar ultraviolet (UV) light and energetic particles precipitating from the magnetosphere produce ionization increasing with altitude. At the same time the particle density is low enough to make the recombination times of the ionized atoms and molecules sufficiently long to allow a significant fraction of the gas to remain ionized. This produces a conducting layer of the atmosphere known as the ionosphere. The ionosphere begins at $\sim\SI{65}{\km}$, has a peak plasma density between 200 and 300 km, and eventually merges with magnetospheric regions $\sim$1000--2000 km altitudes.''

%One hundred kilometers above Earth's surface, give or take, the neutral atmosphere transitions into the conducting ionosphere. Solar ultraviolet radiation is intense enough -- and collisional recombination slow enough -- to maintain a sizable density of charged particles. Moreover, collisions are common enough to disrupt the plasma's frozen-in condition\footnote{In a collisionless plasma, charged particles move freely along magnetic field lines, but cannot cross magnetic field lines; as a result, a compression in the plasma is synonymous to a compression of the plasma that threads it. }, allowing current to flow perpendicular to Earth's dipole magnetic field. 

%In the ionosphere ($\sim\SIrange{e2}{e4}{\km}$), solar ultraviolet radiation is intense enough -- and collisional recombination slow enough -- to maintain a sizable density of charged particles. 

%The effects of mean molecular mass on conductivity are computed per the usual definitions. 
%\begin{align}
%  \sp &= \displaystyle\sum_s \frac{n_s q_s^2}{m_s} \frac{\nu_s}{\nu_s^2 + \Omega_s^2} &
%  \sh &= -\displaystyle\sum_s \frac{n_s q_s^2}{m_s} \frac{\Omega_s}{\nu_s^2 + \Omega_s^2} &
%  \sz &= \displaystyle\sum_s \frac{n_s q_s^2}{m_s \nu_s}
%\end{align}

%\begin{longtable}{ @{\extracolsep{\fill}} lrrr @{\extracolsep{\fill}} }
%  \caption[Integrated Ionospheric Conductivity]{Integrated Ionospheric Conductivity (\si{\S})}
%  \label{tab_sigma_ionos} \\
%  \toprule
%  & $\Sigma_0$ & $\Sigma_P$ & $\Sigma_H$ \\
%  \midrule
%  \endfirsthead
%  \bottomrule
%  \endlastfoot
%  Active Day   & --- & 13.0 & 17.0 \\
%  Quiet Day    & --- & 5.6  & 10.2 \\
%  Active Night & --- & 0.8  & 0.3 \\
%  Quiet Night  & --- & 0.2  & 0.3 \\
%\end{longtable}


