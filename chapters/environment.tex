


\chapter{The Near-Earth Environment}
  \label{ch_environment}

From Earth's surface to a height of a hundred kilometers or so, the atmosphere is a well-behaved fluid: a collisional ensemble of neutral atoms. However, beyond there, its behavior changes dramatically. As altitude increases, solar ultraviolet radiation becomes more intense, which ionizes atmospheric atoms. Density also decreases, slowing collisional recombination. Whereas the neutral atmosphere is held against Earth's surface by gravity, the motion of charged particles is dominated by Earth's geomagnetic field... and the electromagnetic disturbances created as that field is hammered by the solar wind. 

%\todo{Neutral density is larger than charged particle density (?) but it doesn't matter because mean free path is huge. }

The present section outlines the structure of the magnetosphere; that is, the region of space governed primarily by Earth's magnetic field. Particular emphasis is placed on structures which relate closely to field line resonance. 

% -----------------------------------------------------------------------------
% -----------------------------------------------------------------------------
% -----------------------------------------------------------------------------
\section{The Outer Magnetosphere}

Plasma behavior within Earth's magnetosphere is ultimately driven by the solar wind: a hot (\about\SI{100}{\eV}), fast-moving (\about\SI{100}{\km/\s}) plasma threaded by the inerplanetary magnetic field (\about\SI{1}{\nT})\footnote{Listed values correspond to the solar wind at Earth's orbit. }. The density of the solar wind is on the order of \SI{10}{\percc}; in a laboratory setting, this would constitute an ultra-high vacuum (atmospheric density at sea level is \about\SI{e19}{\percc}), but compared to much of the magnetopause it's quite dense. 

The magnetosphere's outer boundary represents a balance between the solar wind dynamic pressure and the magnetic pressure of Earth's dipole field. On the dayside, the dipole is compressed, pushing this boundary to within about \SI{10}{\RE} of Earth\footnote{Distances in the magnetosphere are typically measured in units of Earth radii: $\SI{1}{\RE} \equiv \SI{6378}{\km}$. }. The nightside magnetosphere is stretched into a long tail which may exceed \SI{50}{\RE} in width and \SI{100}{\RE} in length. 

When the interplanetary magnetic field opposes the magnetospheric magnetic field at the nose of the magnetosphere, magnetic reconnection occurs. Closed magnetospheric field lines ``break,'' opening up to the interplanetary magnetic field\footnote{Closed field lines connect at both ends to the magnetic dynamo at Earth's core. Open field lines meet Earth at only one end; the other connects to the interplanetary magnetic field. The distinction is important because charged particle motion is guided by magnetic field lines, as discussed in \cref{ch_flrs}. }. They then move tailward across the poles, dragging their frozen-in plasma with them\footnote{In the outer magnetosphere (as well as most of the inner magnetosphere), collisions are so infrequent that magnetic flux is said to be ``frozen in'' to the plasma. Charged particles move freely along magnetic field lines, but cannot cross from one line to another. Compression of the magnetic field is synonymous with compression of the ambient plasma. See \cref{sec_ionos}. }. Reconnection in the tail allows magnetic field lines to convect back to the day side, across the flanks; this process is called the Dungey cycle\cite{dungey_1961}. 

\todo{Jets from magnetic reconnection... release of magnetic tension! }

Consistent with \amplaw, the interplanetary magnetic field is separated from the magnetosphere by a current sheet: the magnetopause. On the dayside, the magnetopause current flows duskward; on the nightside, it flows dawnward around the magnetotail. 

Plasma within the tail is cool (\about\SI{100}{\eV}) and rarefied (\about\SI{e-2}{\percc}). Earth's dipole is significantly deformed in the magnetotail; field lines in the northern lobe of the tail points more or less Earthward, and vice versa. The two lobes are divided by the plasma sheet, which is comparably hot (\about\SI{1}{\keV}) and dense (\about\SI{1}{\percc}). The plasma sheet carries a duskward current which connects to the magnetopause current. 

% -----------------------------------------------------------------------------
% -----------------------------------------------------------------------------
% -----------------------------------------------------------------------------
\section{The Inner Magnetosphere}

Within about \todo{ $L\sim10$ }\footnote{The McIlwain parameter $L$ is a spatial parameter used to index field lines in Earth's dipole geometry: $L \equiv \frac{r}{\sin^2\theta}$ for colatitude $\theta$ and radius $r$ in Earth radii. For example, the $L=5$ field line passes through the equatorial plane at a geocentric radius of \SI{5}{\RE}, then meets the Earth at a colatitude of $\arcsin \sqrt{ \frac{1}{5} } \sim \SI{27}{\degree}$ (equally, a latitude of $\arcsin \sqrt{ \frac{1}{5} } \sim \SI{63}{\degree}$). }, the dipole magnetic field is not appreciably deformed by the solar wind. As a result, the structures in the inner magnetosphere follow closely from the motion of charged particles in an ideal dipole field. 

The plasmasphere -- a cold (\about\SI{1}{\eV}), dense (\SIrange{e2}{e4}{\percc}) torus of corotating plasma -- is formed by the outward drift of atmospheric ions along magnetic closed field lines. Its outer boundary, is thought to represent a balance between the corotation electric field (per the rotation of Earth's magnetic dipole) and the convection electric field (associated with the convection of magnetic flux during the Dungey cycle). Particle density drops sharply at the edge of the plasmasphere; the boundary is called the plasmapause. The plasmapause typically falls around $L=4$, though during prolonged quiet times it can extend to $L=6$ or larger. 

Energetic particles trapped within the inner magnetosphere are divided into two populations. 

The Van Allen radiation belts are made up of particles with energy above \SI{e5}{\eV} or so. The inner belt ($L\lesssim2$) is primarily composed of protons, the decay remnants of neutrons freed from the atmosphere by cosmic rays. The outer belt ($L\gtrsim4$) is primarily composed of high-energy electrons. 

Particles with energies of (\SIrange{e3}{e5}{\eV}) make up the ring current, which extends more or less from $L\sim3$ to $L\sim5$. Gradient-curvature drift carries ions and electrons in opposite directions; the net result is a westward current. During quiet times, the ring current causes magnetic a southward magnetic field on the order of \SI{1}{\nT} at Earth's equator\footnote{For comparison, Earth's dipole field points north at the equator with a magnitude over \SI{e4}{\nT}. }. 

% -----------------------------------------------------------------------------
% -----------------------------------------------------------------------------
% -----------------------------------------------------------------------------
\section{The Ionosphere}
  \label{sec_ionos}

Earth's neutral atmosphere is an excellent insulator. Collisions are so frequent that charged particles quickly thermalize and recombine. The breakdown of air molecules into a conductive plasma (as happens during a lightning strike, for example) requires electric fields on the order of \SI{e9}{\mV/\m}. 

Currents are also suppressed by the magnetosphere. In the absence of collisions, electrons and ions drift alongside one another in response to an electric field, creating no net current perpendicular to the magnetic field\footnote{The so-called $E$-cross-$B$ drift is associated with a velocity of $\vec{v} = \frac{1}{B^2} \vec{E} \times \vec{B}$, independent of a charged particle's mass or sign. }. Magnetic field lines are (to a good approximation) equipotential contours; electric fields do not form along them to drive parallel currents. 

The ionosphere is a sweet spot between the two regimes. Collisions are frequent enough to disrupt the drift of ions, but not frequent enough to immobilize the electrons. The result is nonzero Pedersen and Hall conductivity, corresponding to current along the electric field and in the $-\vec{E} \times \vec{B}$ direction respectively. Collisions in the ionosphere also give rise to a finite parallel conductivity, allowing for the formation of potential structures along the magnetic field line. 

\todo{Currents in the ionosphere create magnetic fields on the ground. }

\todo{Perpendicular currents close field-aligned currents connecting to the magnetopause or partial ring current. }

\todo{Two-cell convection. }

\todo{Field-aligned acceleration gives rise to precipitation of energetic particles into the atmosphere. }




%From \cite{paschmann_2003}: ``In the thermosphere, the solar ultraviolet (UV) light and energetic particles precipitating from the magnetosphere produce ionization increasing with altitude. At the same time the particle density is low enough to make the recombination times of the ionized atoms and molecules sufficiently long to allow a significant fraction of the gas to remain ionized. This produces a conducting layer of the atmosphere known as the ionosphere. The ionosphere begins at $\sim\SI{65}{\km}$, has a peak plasma density between 200 and 300 km, and eventually merges with magnetospheric regions $\sim$1000--2000 km altitudes.''

%The effects of mean molecular mass on conductivity are computed per the usual definitions. 
%\begin{align}
%  \sp &= \displaystyle\sum_s \frac{n_s q_s^2}{m_s} \frac{\nu_s}{\nu_s^2 + \Omega_s^2} &
%  \sh &= -\displaystyle\sum_s \frac{n_s q_s^2}{m_s} \frac{\Omega_s}{\nu_s^2 + \Omega_s^2} &
%  \sz &= \displaystyle\sum_s \frac{n_s q_s^2}{m_s \nu_s}
%\end{align}

% -----------------------------------------------------------------------------
% -----------------------------------------------------------------------------
% -----------------------------------------------------------------------------
\section{Geomagnetic Storms and Substorms}
  \label{sec_storms}

The quiet geomagnetic behavior described in \cref{ch_environment} is periodically disturbed by transient solar phenomena. 

\todo{Coronal mass ejection. Corotating interaction region. }

The present section discusses the magnetospheric response to such events. 

\todo{Storm index. Dst, but also mention Sym-H. }

\todo{Enhanced reconnection. }

\todo{Definition of a substorm comes from \cite{akasofu_1964}. Revised by \cite{mcpherron_1973_substorms}. }

\todo{Precipitation into the atmosphere. }

\todo{Controversy around substorm onset. }













%\begin{longtable}{ @{\extracolsep{\fill}} lrrr @{\extracolsep{\fill}} }
%  \caption[Integrated Ionospheric Conductivity]{Integrated Ionospheric Conductivity (\si{\S})}
%  \label{tab_sigma_ionos} \\
%  \toprule
%  & $\Sigma_0$ & $\Sigma_P$ & $\Sigma_H$ \\
%  \midrule
%  \endfirsthead
%  \bottomrule
%  \endlastfoot
%  Active Day   & --- & 13.0 & 17.0 \\
%  Quiet Day    & --- & 5.6  & 10.2 \\
%  Active Night & --- & 0.8  & 0.3 \\
%  Quiet Night  & --- & 0.2  & 0.3 \\
%\end{longtable}


