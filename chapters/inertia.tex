% %%%%%%%%%%%%%%%%%%%%%%%%%%%%%%%%%%%%%%%%%%%%%%%%%%%%%%%%%%%%%%%%%%%%%%%%%%%%%
% %%%%%%%%%%%%%%%%%%%%%%%%%%%%%%%%%%%%%%%%%%%%%%%%%% Results of Numerical Model
% %%%%%%%%%%%%%%%%%%%%%%%%%%%%%%%%%%%%%%%%%%%%%%%%%%%%%%%%%%%%%%%%%%%%%%%%%%%%%

\chapter{Electron Inertial Effects}
\label{ch_inertia}

\todo{Bob mentions something about electron inertia or pressure being in his 2011 paper. }

The model described in \cref{ch_model} has the notable omission of parallel electric fields and parallel currents. That situation can be remedied by the addition of the electron inertial term in \ohmlaw. 

Old parallel electric field formulation. Recall $\vec{F} \equiv \curl{B}$. 
\begin{align}
  \ez \ddt E_\parallel &= \frac{1}{\mu_0} F_\parallel - \sz E_\parallel
\end{align}

New parallel electric field formulation. The parallel current must now be tracked explicitly. 
\begin{align}
  \label{def_inertia}
  \ez \ddt E_\parallel &= \frac{1}{\mu_0} F_\parallel - J_\parallel &
  \ddt J_\parallel &= \frac{n e^2}{m} E_\parallel - \nu J_\parallel
\end{align}

In the new formulation, $J_\parallel$ (equally, $J_3$) is solved with integrating factors and $E_\parallel$ ($E_3$) can be advanced directly. 
\begin{align}
  \begin{split}
  E_3 &\assign E_3 + c^2 \dt \, \lr{ g_{31} F^1 + g_{33} F^3 } - \frac{\dt}{\ez} J_3 \\
  J_3 &\assign J_3 \exp \arg{ - \nu \dt } + \frac{n e^2}{m} \dt \, E_3 \exp \arg{ -\nu \tfrac{\dt}{2} }
  \end{split}
\end{align}

Recall that the electric and magnetic fields are staggered by half a time step. The current is defined with the magnetic fields, offset from the electric fields. 

% =============================================================================
% =============================================================================
% =============================================================================
\section{The Boris Approximation}

Note that 
\begin{align}
  \ddt E_\parallel &\sim -\frac{1}{\ez} J_\parallel &
  & \text{and} & 
  \ddt J_\parallel &\sim \frac{n e^2}{\me} E_\parallel &
  & \text{so} &
  \frac{ \partial^2 }{ \partial t^2 } E_\parallel &\sim -\op^2 E_\parallel
\end{align}

That is, the addition of the electron inertial term in \ohmlaw allows plasma oscillations. 

As noted in \cref{sec_e}, the plasma frequency is very large. Much larger than $\frac{1}{\dt}$. But $\op \dt < 1$ is necessary for stability. In order to accommodate that condition, the time step in some runs would need to be dropped by three orders of magnitude; a simulation slated for one hour would suddenly take six weeks to complete. 

The time step dictated by the \Alfven speed and grid spacing is typically on the order of $\SI{10}{\us}$, while the plasma frequency can be as small as $\SI{10}{\ns}$. 

But there is another way into Mordor. There's a path, and some stairs and then a tunnel. 

The plasma frequency (and the speed of light) can be decreased by taking an artificially large value for \ez. 

Such approximations have been staples of numerical MHD models since Boris' work in 1970\cite{boris_1970}.

Lysak and Song\cite{lysak_2001} demonstrate the validity of such an approximation. To paraphrase their work, take \cref{def_inertia} and suppose that $E_\parallel$ and $J_\parallel$ are oscillating at a frequency $\omega$. Then,
\begin{align}
  - i \omega \ez E_\parallel &= \oomz F_\parallel - J_\parallel & - i \omega J_\parallel &= \frac{n e^2}{\me} E_\parallel - \nu J_\parallel
\end{align}

So
\begin{align}
  \label{boris_criterion}
  \lr{ 1 - \frac{\omega^2 - i \nu \omega}{\op^2} } E_\parallel &= \frac{c^2}{\op^2} \lr{ \nu - i \omega } F_\parallel
\end{align}

Here $\frac{c}{\op}$ is the electron inertial length. While the speed of light and the plasma frequency each depend on \ez, their ratio does not. So long as $\lr{ 1 - \frac{\omega^2 - i \nu \omega}{\op^2} } \sim 1$, a change in \ez should not affect model behavior. 

For the purposes of simulating ultra low frequency waves, \cref{boris_criterion} allows perhaps-implausibly large Boris factors; even increasing \ez by a factor of \num{e6} gives $\left| \frac{\omega^2 + i \omega \nu}{ \op^2 } \right| \lesssim 0.01$. At that point, in some places, the speed of light is significantly slower than the \Alfven speed. 

\todo{Ronnmark\cite{ronnmark_2000} calls this ``anisotropic vacuum.'' }

\todo{Show a plot of these frequency ratios in the ionosphere? }

\todo{Generalized \ohmlaw, in case we decide we need it. Could talk through why all of the other terms are OK to neglect.  
\begin{align}
  \vec{E} + \cross{U}{B} & = 
  \eta \vec{J} + \tfrac{\me}{n e^2} \lrb{
    \tfrac{\partial}{\partial t} \vec{J} + \nabla \cdot \lr{ 
      \vec{J} \, \vec{U} + \vec{U} \,\vec{J} +
      \tfrac{1}{n e} \vec{J} \, \vec{J} } } +
  \tfrac{1}{n e} \cross{J}{B} -
  \tfrac{1}{n e} \div{ \vec{P_e} }
\end{align}
}


% =============================================================================
% =============================================================================
% =============================================================================
\section{Effect on the Simulation}


\begin{figure}[H]
    \centering
    \includegraphics[width=\textwidth]{figures/B_1_004_016mHz.pdf}
    \caption[Magnetic Field Comparison With and Without Electron Inertial Effects]{}
    \label{fig_B_1_004_016mHz}
\end{figure}



% =============================================================================
% =============================================================================
% =============================================================================
\section{Field-Aligned Current}

\begin{figure}[H]
    \centering
    \includegraphics[width=\textwidth]{figures/JS_1_16mHz.pdf}
    \caption[Field-Aligned Current and the Toroidal Mode: Active Day]{}
    \label{fig_JS_1_16mHz}
\end{figure}

% =============================================================================
% =============================================================================
% =============================================================================
\section{Inertial Length Scales}


