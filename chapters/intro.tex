


\chapter{Introduction}
  \label{ch_intro}

%ALEX NOTES: 
%- Should decide if you're going to use contractions or not; many would say not to in formal writing

\todo{In 1859, humanity was working hard to get its shit together.} The United States moved steadily toward the American Civil War, which would abolish slavery and consolidate the power of the federal government. A slew of conflicts in Southern Europe, such as the Austro-Sardinian War, set the stage for the unification of Italy. The Taiping Civil War --- one of the bloodiest conflicts of all time --- is considered by many to mark the beginning of modern Chinese history. Origin of Species was published. The first transatlantic telegraph cable was laid.

Meanwhile, ambivalent to humanity, the Sun belched an intense burst of charged particles and magnetic energy directly at Earth. The resulting geomagnetic storm\footnote{The Solar Storm of 1859 is also called the Carrington Event, after English astronomer Richard Carrington. He drew a connection between the storm's geomagnetic activity and the sunspots he had observed the day before.} caused telegraph systems to fail across the Western hemisphere, electrocuting operators and starting fires\cite{green_2006,tsurutani_2003}. Displays of the northern lights were visible as far south as Cuba. 

The Solar Storm of 1859 was perhaps the most powerful in recorded history, but by no means was it a one-time event. The Sun discharges hundreds of coronal mass ejections (CMEs) per year, of all sizes, in all directions. In fact, a comparably large CME narrowly missed Earth in 2012\cite{nasa_2012}. Had it not, it's estimated it would have caused widespread, long-term electrical outages, with a damage toll on the order of \num{e12} dollars\cite{lloyds_2013}. 

The Sun's extreme --- and temperamental --- effect on Earth's magnetic environment makes a compelling case for the ongoing study of space weather. Such research has evolved over the past century from sunspot counts and compass readings to multi-satellite missions and supercomputer simulations. Modern methods have dramatically increased humanity's understanding of the relationship between the Sun and the Earth; however, significant uncertainty continues to surround geomagnetic storms, substorms, and the various energy transport mechanisms that make them up. 

The present work focuses in particular on the phenomenon of field line resonance: \Alfven waves bouncing between the northern and southern hemispheres. Such waves play an important part in the energization of magnetospheric particles, the transport of energy from high to low altitude, and the driving of currents at the top of the atmosphere. 

\section{Structure of the Present Work}

\todo{Make this section read nicely. }

\cref{ch_environment} briefly introduces the near-Earth environment. 

\cref{ch_flrs} surveys the field line resonance phenomenon in terms of its underlying physics and notable theoretical and observational work. It also indicates several open questions pertaining to field line resonances. 

\cref{ch_math} explores the fundamental equations of electromagnetic waves in a cold, resistive plasma (such as Earth's magnetosphere). 

\cref{ch_model} presents a new two and a half dimensional numerical model for the simulation of field line resonances in the inner magnetosphere, particularly those with large azimuthal modenumber. 

\cref{ch_inertia} shows how electron inertial effects can be added to the model, touches on what can be learned from them, and explains why they are not presently feasible. 

\cref{ch_results} describes the core numerical results of the work, unifying several of the questions posed in \cref{ch_flrs}. 

\cref{ch_rbsp} puts the numerical results in physical context through the analysis of data from the Van Allen Probes mission. 

\cref{ch_conclusion} briefly summarizes the present work and suggests future directions. 

%\todo{The storm of 1859 presented compelling evidence that the Sun drives geomagnetic activity. In the decades that followed, a model took shape to describe the mechanisms of energy transfer between the Sun and the Earth. Based on auroral observations and data from ground-based magnetometers, Birkeland argued for the existence of a constant outflow of electrons and ions from the Sun -- the solar wind. The advent of high-frequency radio communication allowed Kennelly, Heaviside, and others to probe the electrical properties of the upper atmosphere. }

%\todo{The study of space weather was revolutionized by the development of sounding rockets and satellites in the mid twentieth century. This allowed direct observation of the structure of the near-Earth environment, including, crucially, the waves that carry energy through it. }

%\todo{Not least among these advances was the discovery of \Alfven waves. {\Alfven}ic aurora. Carry energy and particles. }

%The study of space weather revolves around the transfer of energy from the Sun to the Earth. Ultra low frequency waves in particular are an important energy transport mechanism between the magnetosphere's outer boundary (at the solar wind) and its inner boundary (at the top of the atmosphere). 

%\footnote{Uppercase \X, \Y, and \Z are used to indicate GSE coordinates: \Xhat points from the Earth to the Sun; \Yhat is perpendicular to \Xhat in the Sun's ecliptic plane, pointing duskwards; \Zhat points north, out of the ecliptic plane. In later chapters, lowercase \x, \y, and \z are used to define a more or less analogous corodinate system with respect to Earth. }

% The moon is at \SI{60}{\RE}. 

% Transient solar wind phenomena, such as coronal mass ejections, are also known to be related to geomagnetic disturbances at Earth. Jesse cites here: 

% R. L. McPherron. Physical processes producing magnetospheric substorms and mangetic storms. In J. A. Jacobs, editor, Geomagnetism, volume 4, chapter 7. Academic Press, 1991.

% G. Rostoker. Substorms. In Handbook of the Solar-Terrestrial Environment, chapter 15. Springer-Verlag, 2007.

% M. Shulz. Magnetospheres. In Handbook of the Solar-Terrestrial Environment, chapter 7. Springer-Verlag, 2007.

% papers mentioned during Yan's talk. mostly about alfven acceleration and nonlinear effects. 
% Vasyliunas 1970, 1984
% Hasegawa 1976
% Goertz 1991
% Stasiewicz et al 2000
% Haerendel 2008
% Song & Lysak 1994, 1999, 2000, 2001, 2006, 2011, 2012
% Inverted V?
% Double layers? 
% Charge holes? 

