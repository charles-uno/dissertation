% %%%%%%%%%%%%%%%%%%%%%%%%%%%%%%%%%%%%%%%%%%%%%%%%%%%%%%%%%%%%%%%%%%%%%%%%%%%%%
% %%%%%%%%%%%%%%%%%%%%%%%%%%%%%%%%%%%%%%%% Survey of the Near-Earth Environment
% %%%%%%%%%%%%%%%%%%%%%%%%%%%%%%%%%%%%%%%%%%%%%%%%%%%%%%%%%%%%%%%%%%%%%%%%%%%%%

\chapter{The Near-Earth Environment}
\label{ch_intro}

\todo{So far, this chapter is just a tentative outline. }

One hundred kilometers above Earth's surface, more or less, the neutral atmosphere transitions sharply into the conducting ionosphere.

300

1000

3000

10,000 (2RE geocentric)

30,000 (5 RE)






Ionosphere is dense enough to be shaped by gravity, but as altitude increases, magnetic field dominates.

The nitrogen density (w at sea level) drops from x at 100km to y at 1000km to z at 10**4 km.

By 10**4 km, the curvature of Earth's dipole magnetic field becomes apparent... and, between the increasing ion abundance and the decreasing strength of gravity, it dominates particle behavior.

At 10**5 km, the bow shock. On the dayside, at least. Balance between Earth's magnetic field and that of the sun. Another current sheet... This one far more intense.

Still less than halfway to the moon. That's 3e5 km.

This all takes place well within the orbit of the moon. Moon is about 60 RE away

Where are satellites? Geosynthronous? 

This with is concerned with the behavior of electromagnetic waves that propagate inside the magnetosheath, but outside the ionosphere; in fact, they play a significant role in the transport of energy from the former to the latter. 

Free electron density...

Still mostly neutrals, but collisions are so rare that they don't matter. At x, the mean free path of a neutral atom is comparable to... 

%There are a lot of interrelated things going on, so it's hard to describe Earth's environment one step at a time. Look at Scott's thesis -- he did this well, right? 

%Heliosphere, Magnetosphere, Ionosphere, Atmosphere?

%Current systems, convective systems, density profiles? 

%\begin{figure}
%  \includegraphics[width=5.75in, height=2in]{figures/image.jpg}
%  \caption{Lorem ipsum dolor sit amet, consectetur adipiscing elit, sed do eiusmod tempor incididunt ut labore et dolore magna aliqua.}
%  \label{fig_test}
%\end{figure}

It's all about energy transfer! Sun generates energy through nuclear reactions. Some of this energy is transported in the solar wind, which drives behavior in the near-Earth environment. 

%Typical solar wind density is $\sim$ \SI{5}{/\cm\cubed}. Typical solar wind velocity at Earth is \SIrange{e2}{e3}{\km/\s}. Typical solar wind particle energy is \SIrange{1}{10}{\kilo\eV}. Density can vary by $\sim$3 orders of magnitude, and velocity by one, during times of high solar activity. CMEs can also mess with the north/south component of the interplanetary magnetic field. 

At Earth's orbit, the solar magnetic field makes more-or-less a \SI{45}{\degree} angle with the \X axis. 
\footnote{Uppercase \X, \Y, and \Z are used to indicate GSE coordinates: \X points from the Earth to the Sun; \Y is perpendicular to \X in the Sun's ecliptic plane, pointing duskwards; \Z points north, out of the ecliptic plane. In later chapters, lowercase \x, \y, and \z are used to define a more-or-less analogous corodinate system with respect to Earth. }

% Solar wind is what deforms Earth's magnetic field to form the magnetosphere. 

% Transient solar wind phenomena, such as coronal mass ejections, are also known to be related to geomagnetic disturbances at Earth. Jesse cites here: 

% R. L. McPherron. Physical processes producing magnetospheric substorms and mangetic storms. In J. A. Jacobs, editor, Geomagnetism, volume 4, chapter 7. Academic Press, 1991.

% G. Rostoker. Substorms. In Handbook of the Solar-Terrestrial Environment, chapter 15. Springer-Verlag, 2007.

% This might just be worth tracking down... Jesse cites several chapters: 

% M. Shulz. Magnetospheres. In Handbook of the Solar-Terrestrial Environment, chapter 7. Springer-Verlag, 2007.

% papers mentioned during Yan's talk. mostly about alfven acceleration and nonlinear effects. 
% Vasyliunas 1970, 1984
% Hasegawa 1976
% Goertz 1991
% Stasiewicz et al 2000
% Haerendel 2008
% Song & Lysak 1994, 1999, 2000, 2001, 2006, 2011, 2012
% Inverted V?
% Double layers? 
% Charge holes? 

% =============================================================================
% =============================================================================
% =============================================================================
\section{The Outer Magnetosphere}

The outer magnetosphere is a region where the field lines are closed, but significantly deformed by the solar wind. 

\subsection{The Magnetopause}

\subsection{The Magnetotail}

\subsection{Cusp Regions}

\todo{The cusp regions might not even need to be mentioned... they're not specifically important here. }

% =============================================================================
% =============================================================================
% =============================================================================
\section{The Inner Magnetosphere}

In the inner magnetosphere, field lines are closed, and are approximately dipolar. 

\subsection{The Plasmasphere}

\subsection{Ring Currents}

\subsection{The Radiation Belts}

% =============================================================================
% =============================================================================
% =============================================================================
\section{The Ionosphere}

The ionosphere is immediately above Earth's neutral atmosphere. 

%\todo{At present, ionospheric conductivity profiles and the \Alfven speed aren't really illustrated until \cref{sec_ionos}. That's pretty awkward, since \cref{ch_math} makes extensive use of those quantities in a bunch of dispersion relations. They should be introduced before they're used! }

\subsection{Field-Aligned Currents}

\subsection{Pedersen and Hall Currents}

\subsection{Ionospheric Stratification}

% =============================================================================
% =============================================================================
% =============================================================================
\section{Geomagnetic Disturbances}

\subsection{Storms}

\subsection{Substorms}






