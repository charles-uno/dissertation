% %%%%%%%%%%%%%%%%%%%%%%%%%%%%%%%%%%%%%%%%%%%%%%%%%%%%%%%%%%%%%%%%%%%%%%%%%%%%%
% %%%%%%%%%%%%%%%%%%%%%%%%%%%%%%%%%%%% Compute the Dispersion Relation for Tuna
% %%%%%%%%%%%%%%%%%%%%%%%%%%%%%%%%%%%%%%%%%%%%%%%%%%%%%%%%%%%%%%%%%%%%%%%%%%%%%

\chapter{Waves in Cold Resistive Plasmas}
  \label{ch_math}

\todo{Sketch out what a dispersion relation is, why it works, why it's interesting. }

\todo{Explain that the inclusion of conductivity is novel. }

\todo{This chapter works out the sorts of waves that might be expected in the numerical model. It starts with the same equations that are used by the model -- Maxwell's equations and \ohmlaw. The resulting dispersion relation is too high-ordered for a direct solution, so several limits of interest are considered. } 

\todo{At this end of this chapter, there will be a discussion of what is specifically interesting about the findings. That doesn't really exist yet. Or... actually, should we discuss interesting things as we find them? Note that we find $\omega^2 = k^2 \va^2$ several times. }

Cold, linearlized \amplaw and \farlaw. The vector $\vec{B}$ is the perturbation to the zeroth-order magnetic field. 
\begin{align}
  \ddt \vec{B} &= -\curl{E} & \tensor{\epsilon} \cdot \ddt \vec{E} &= \oomz \curl{B} - \vec{J}
\end{align}

\ohmlaw. Electron inertial effects are included in the parallel direction. See \cref{ch_inertia}. 
\begin{align}
  \frac{\me}{n e^2} \ddt J_\parallel & = 
    \sz E_\parallel - J_\parallel &
  0 & = 
    \tensor{\sigma}_\bot \cdot \vec{E}_\bot - \vec{J}_\bot
\end{align}

Suppose that the fields and currents are resonating as $\exp \arg{i \vec{k} \cdot \vec{x} - i \omega t }$. Evaluate the derivatives. Eliminate magnetic fields and currents. 
\begin{align}
  \label{disp_setup}
\begin{alignedat}{4}
%  \label{disp_para}
  & 0 = E_\parallel && + \frac{c^2}{\omega^2} \lr{ \vec{k} \, \vec{k} \cdot \vec{E} - k^2 \vec{E} }_\parallel && + \frac{i \op^2 }{\omega \lr{\nu - i \omega} } && E_\parallel \\
%  \label{disp_perp}
  & 0 = \vec{E}_\bot && + \frac{\va^2}{\omega^2} \lr{ \vec{k} \, \vec{k} \cdot \vec{E} - k^2 \vec{E} }_\bot && + \frac{i}{\ep \omega} \tensor{\sigma}_\bot \cdot && \vec{E}_\bot
\end{alignedat}
\end{align}

The above expression makes use of the vector identity $\vec{k} \times \vec{k} \times \vec{E} = \vec{k} \, \vec{k} \cdot \vec{E} - k^2 \vec{E}$. The \Alfven speed, speed of light, plasma frequency, and parallel conductivity are defined in the usual way: 
\begin{align}
  \label{def_basics}
  \va^2 & \equiv \frac{1}{\mz \ep} &
  c^2 & \equiv \frac{1}{\mz \ez} &
  \op^2 & \equiv \frac{n e^2}{\me \ez} &
  \sz & \equiv \frac{n e^2}{\me \nu}
\end{align}

Note that this definition of the \Alfven speed takes into account the displacement current correction which is important when \va approaches $c$. 

\cref{disp_setup} is then assembled into the usual dispersion tensor form:
\begin{align}
  \label{disp}
  0 &= \lr{ \tensor{ \mathbb{I} } + \frac{1}{\omega^2} \tensor{V}^2 \cdot \vec{k} \, \vec{k} - \frac{k^2}{\omega^2} \tensor{V}^2 + \frac{i}{\omega} \tensor{\Omega} } \cdot \vec{E}
\end{align}

Where $\tensor{ \mathbb{I} }$ is the identity tensor and in \x-\y-\z coordinates, 
\begin{align}
  \tensor{V}^2 &\equiv 
    \mmm{\va^2}{0}{0}
        {0}{\va^2}{0}
        {0}{0}{c^2} &
  & \text{and} &
  \tensor{\Omega} &\equiv 
    \mmm{ \frac{\sp}{\ep} }{ \frac{-\sh}{\ep} }{0}
        { \frac{\sh}{\ep} }{ \frac{\sp}{\ep} }{0}
        {0}{0}{ \frac{\op^2}{\lr{\nu - i \omega}} } 
\end{align}

In \cref{disp}, the expression in parentheses is the dispersion tensor, $\dispersiontensor$. Nontrivial solutions exist only when $\det \dispersiontensor = 0$. This gives rise to a seventh-order polynomial in $\omega$, so rather than a direct solution it's necessary to consider limits of specific interest. 

%\begin{align}
%  \dispersiontensor & = \tensor{ \mathbb{I} }
%                      + \frac{k^2}{\omega^2} 
%                        \mmm{-\va^2 \cos^2 \theta}{0}{\va^2 \sin \theta \cos \theta}
%                            {0}{-\va^2}{0}
%                            {c^2 \sin \theta \cos \theta}{0}{-c^2 \sin^2 \theta}
%                      + \frac{i}{\omega}
%                        \mmm{ \frac{\sp}{\ep} }{-\frac{\sh}{\ep} }{0}
%                            { \frac{\sh}{\ep} }{ \frac{\sp}{\ep} }{0}
%                            {0}{0}{ \frac{\op^2}{\nu - i \omega} }
%\end{align}

% =============================================================================
% =============================================================================
% =============================================================================
\section{Parallel or Perpendicular Propagation}
  \label{sec_par_perp}

\todo{Parallel propagation is ineresting because it's a naive representation of a field line resonance. Perpendicular propagation is interesting because the movement of energy across field lines is a topic of particular interest. In both cases, the cross terms in $\vec{k} \, \vec{k}$ vanish, decoupling the parallel and perpendicular polarizations. }

\todo{Without loss of generality, the wave vector $\vec{k}$ is presumed to lie in the \x-\z plane. The distinction between the azimuthal and crosswise directions is revisited in \cref{sec_math_implications}. }

\todo{ACTUALLY: does this even matter? If we keep $\phi$ in there in addition to $\theta$, does it cancel out? Seems like it should. }

% -----------------------------------------------------------------------------
% -----------------------------------------------------------------------------
% -----------------------------------------------------------------------------
\subsection{Parallel Polarization}

In the parallel and perpendicular propagation limits respectively, the parallel factors of the determinants of \dispersiontensor are:
\begin{align}
  \label{parallel_math_setup}
  \begin{split}
  % Parallel propagation, parallel polarization. 
  0 &= \omega^2 + i \nu \omega - \op^2 \\
  % Perpendicular propagation, parallel polarization. 
  0 &= \omega^3 + i \nu \omega^2
  - \lr{ k^2 c^2 + \op^2 } \omega
  - i k^2 c^2 \nu
  \end{split}
\end{align}

Both expressions in \cref{parallel_math_setup} can be solved directly, though the solution to the (cubic) perpendicular propagation case is too long to be useful. After expanding with respect to a very large plasma frequency, the solutions can be written (for parallel and perpendicular propagation respectively): 
\begin{align}
  \label{parallel_math_final}
  \begin{split}
  % Parallel propagation, parallel polarization. 
  \omega^2 & = \op^2 - i \nu \op + ... \\
  % Perpendicular propagation, parallel polarization. 
  \omega^2 & = k^2 c^2 + \op^2 - i \nu \op + ...
  \end{split}
\end{align}

The first expression in \cref{parallel_math_final} -- describing the parallel-polarized component of a parallel-propagating wave -- doesn't describe a wave at all; there's no dependence on the wave vector $k$. Instead, it describes the plasma oscillation. The second expression is the O mode: a compressional wave, oscillating just above the plasma frequency, with a parallel electric field. 

% -----------------------------------------------------------------------------
% -----------------------------------------------------------------------------
% -----------------------------------------------------------------------------
\subsection{Perpendicular Polarization}
  \label{sec_perp}

The perpendicular factors of the dispersion tensor's determinant, in the cases of parallel and perpendicular propagation respectively, are: 
\begin{align}
  \label{perpendicular_math_setup}
  \begin{split}
  % Parallel propagation, perpendicular polarization. 
  0 &= \omega^4 + 2 i \frac{\sp}{\ep} \omega^3
  - \lr{ 2 k^2 \va^2 + \frac{\sp^2 + \sh^2}{\ep^2} } \omega^2
  - 2 i k^2 \va^2 \frac{\sp}{\ep} \omega
  + k^4 \va^4 \\
  % Perpendicular propagation, perpendicular polarization. 
  = &= \omega^3 + 2 i \frac{\sp}{\ep} \omega^2
  - \lr{ 2 k^2 \va^2 + \frac{\sp^2 + \sh^2}{\ep^2} } \omega
   - i k^2 \va^2 \frac{\sp}{\ep}
  \end{split}
\end{align}

First expression of \cref{perpendicular_math_setup} can be solved directly. Noting that $\pm$ and $\opm$ are independent,
\begin{align}
  % Parallel propagation, perpendicular polarization. 
  \omega = \lr{ \frac{\pm \sh - i \sp}{2 \ep} } \opm \sqrt{ k^2 \va^2 + \lr{ \frac{\pm \sh - i \sp}{2 \ep} }^2 }
\end{align}

This wave is propagating along the field line, so the Pedersen and Hall conductivities are small for the vast majority of its trajectory. Expanding, then, gives
\begin{align}
  \label{perp_final_z}
  % Parallel propagation, perpendicular polarization. 
  \omega^2 = k^2 \va^2 \opm k \va \frac{\sh \pm i \sp}{\ep} + ...
\end{align}

The second expression of \cref{perpendicular_math_setup} can also be solved directly, though its roots are impractically long. They are expanded with respect to small conductivities (as would be seen by a perpendicular-propagating wave at high altitude) and at large conductivity (representing the perpendicular-propagating wave within the ionosphere). Respectively, 
\begin{align}
  \label{perp_final_x}
  \begin{split}
  % Perpendicular propagation, perpendicular polarization. Small conductivity. 
  \omega^2 & = k^2 \va^2 \pm i k \va \frac{\sp}{\ep} + ... \\
  % Perpendicular propagation, perpendicular polarization. Large conductivity. 
  \omega^2 & = k^2 \va^2 + \lr{ \frac{\sh \pm i \sp}{\ep} }^2 + ...
  \end{split}
\end{align}

\cref{perp_final_z,perp_final_x} describe \Alfven waves: waves traveling at the \Alfven speed (shifted by the ionospheric conductivity) with electric fields perturbations perpendicular to the zeroth-order magnetic field.  

%\begin{align}
%  \label{parallel_dispersion_tensor}
%  \dispersiontensor_\parallel & = \tensor{ \mathbb{I} }
%                      - \frac{k^2 \va^2}{\omega^2} 
%                        \mmm{1}{0}{0}
%                            {0}{1}{0}
%                            {0}{0}{0}
%                      + \frac{i}{\omega}
%                        \mmm{ \frac{\sp}{\ep} }{-\frac{\sh}{\ep} }{0}
%                            { \frac{\sh}{\ep} }{ \frac{\sp}{\ep} }{0}
%                            {0}{0}{ \frac{\op^2}{\nu - i \omega} }
%\end{align}

%\begin{align}
%  \label{perp_dispersion_tensor}
%  \dispersiontensor_\bot & = \tensor{ \mathbb{I} }
%                      - \frac{k^2}{\omega^2} 
%                        \mmm{0}{0}{0}
%                            {0}{\va^2}{0}
%                            {0}{0}{c^2}
%                      + \frac{i}{\omega}
%                        \mmm{ \frac{\sp}{\ep} }{-\frac{\sh}{\ep} }{0}
%                            { \frac{\sh}{\ep} }{ \frac{\sp}{\ep} }{0}
%                            {0}{0}{ \frac{\op^2}{\nu - i \omega} }
%\end{align}

% =============================================================================
% =============================================================================
% =============================================================================
\section{High Altitude Limit}
  \label{sec_high_alt}

In the limit of large radial distance, where the density is very low, it's reasonable to neglect the Pedersen conductivity, the Hall conductivity, and the collision rate. 

Without loss of generality, the wave vector $\vec{k}$ can be presumed to lie in the \x-\z plane\footnote{The azimuthal-propagation case is discussed in \cref{sec_math_implications}}. 

% -----------------------------------------------------------------------------
% -----------------------------------------------------------------------------
% -----------------------------------------------------------------------------
\subsection{Azimuthal Polarization}

Whereas in \cref{sec_par_perp} the parallel component of the determinant of the dispersion tensor could be factored out, the high altitude limit decouples the azimuthal component from those in the meridional plane. The azimuthal component gives, simply, 
\begin{align}
  \omega^2 & = k^2 \va^2
\end{align}

This solution is analogous to the results in \label{sec_perp}: a wave propagating at the \Alfven speed, with electric field perturbation perpendicular to both the zeroth-order magnetic field and the wave vector. 

% -----------------------------------------------------------------------------
% -----------------------------------------------------------------------------
% -----------------------------------------------------------------------------
\subsection{Meridional Polarization}

The meridional components of the determinant of the high-altitude limit of the dispersion tensor give: 
\begin{align}
  \label{high_meridional_setup}
  0 &= \omega^4 
  - \lr{ k_\parallel^2 \va^2 + k_\bot^2 c^2 + \op^2 } \omega^2
  + k_\parallel^2 \va^2 \op^2
\end{align}

Where $k_\parallel$ and $k_\bot$ are the parallel and crosswise components of the wave vector. The a

\label{high_meridional_setup} is quadratic in $\omega^2$. Its solution is
\begin{align}
  \omega^2 & = \frac{1}{2} \lr{ k_\parallel^2 \va^2 + k_\bot^2 c^2 + \op^2 }
  \pm \sqrt{ \frac{1}{4} \lr{ k_\parallel^2 \va^2 + k_\bot^2 c^2 + \op^2 }^2 
    - k_\parallel^2 \va^2 \op^2 }
\end{align}

Noting that $\op$ is very large, the two roots simplify to
\begin{align}
  \label{high_meridional_final}
  \begin{split}
  \omega^2 & = k_\parallel^2 \va^2 + ... \\
  \omega^2 & = k_\parallel^2 \va^2 + k_\bot^2 c^2 + \op^2 + ...
  \end{split}
\end{align}

\todo{The first part looks like an \Alfven wave. The second is faster than the plasma frequency, so we don't really care about it. }

%\begin{align}
%  \label{high_dispersion_tensor}
%  \dispersiontensor_{\infty} & = \tensor{ \mathbb{I} }
%                      - \frac{k^2}{\omega^2} 
%                        \mmm{\va^2 \cos^2 \theta}{0}{-\va^2 \sin \theta \cos \theta}
%                            {0}{\va^2}{0}
%                            {-c^2 \sin \theta \cos \theta}{0}{c^2 \sin^2 \theta}
%                      - \frac{\op^2}{\omega^2}
%                        \mmm{0}{0}{0}
%                            {0}{0}{0}
%                            {0}{0}{1}
%\end{align}


% =============================================================================
% =============================================================================
% =============================================================================
\section{Implications for This Work}
  \label{sec_math_implications}

% -----------------------------------------------------------------------------
% -----------------------------------------------------------------------------
% -----------------------------------------------------------------------------
\subsection{Mode Coupling}

\todo{Have we got enough/appropriate math here to talk about how $\sh$ rotates fields at the ionosphere? Doubtful. }

% -----------------------------------------------------------------------------
% -----------------------------------------------------------------------------
% -----------------------------------------------------------------------------
\subsection{High Modenumber Cutoff}

As $\azm$ becomes large, it puts a lower bound on the modenumber, and thus on the frequency. For an \Alfven wave, $\omega^2 = k^2 \va^2$,
\begin{align}
  k \ge k_\phi = \frac{\azm}{2 \pi r} \qquad \text{so} \qquad \omega \ge \frac{\azm}{2 \pi r} \va
\end{align}

Any wave with a frequency below this threshold will become evanescent. 

\todo{This point is pretty important. It's revisited in \cref{sec_driving}. }


