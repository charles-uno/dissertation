% %%%%%%%%%%%%%%%%%%%%%%%%%%%%%%%%%%%%%%%%%%%%%%%%%%%%%%%%%%%%%%%%%%%%%%%%%%%%%
% %%%%%%%%%%%%%%%%%%%%%%%%%%%%%%%%%%%%%%%%%%%%%%%%% Work Out the Expected Waves
% %%%%%%%%%%%%%%%%%%%%%%%%%%%%%%%%%%%%%%%%%%%%%%%%%%%%%%%%%%%%%%%%%%%%%%%%%%%%%

\chapter{Dispersion Relation}
\label{math_chapter}

We start with the equations used in the model, as described in \cref{model_equations_section}. 

\begin{align}
  \ddt \vec{B} &= -\curl{E} \\
  \epsilon_\bot \ddt \vec{E}_\bot &= \frac{1}{\mu_0} \left( \curl{B} \right)_\bot - \tensor{\sigma}_\bot \cdot \vec{E}_\bot \\
  \epsilon_\parallel \ddt E_\parallel &= \frac{1}{\mu_0} \left( \curl{B} \right)_\parallel - J_\parallel \\
  \frac{m \sigma_0}{n e^2} \ddt J_\parallel &= \sigma_0 E_\parallel - J_\parallel
\end{align}

We're looking for resonances. If we assume that everything (at least locally) goes as $\exp(i \vec{k} \cdot \vec{x} - i \omega t )$ then what do we find? 

\begin{align}
  - i \omega \vec{B} &= - i \vec{k} \times \vec{E} \\
  - i \omega \epsilon_\bot \vec{E}_\bot &= \frac{1}{\mu_0} \left( - i \vec{k} \times \vec{B} \right)_\bot - \tensor{\sigma}_\bot \cdot \vec{E}_\bot \\
  - i \omega \epsilon_\parallel E_\parallel &= \frac{1}{\mu_0} \left( - i \vec{k} \times \vec{B} \right)_\parallel - J_\parallel \\
  - i \frac{m \omega \sigma_0}{n e^2} J_\parallel &= \sigma_0 E_\parallel - J_\parallel
\end{align}

With a bit of algebra, we can eliminate everything but the electric fields. 

\begin{align}
  0 &= - i \omega \epsilon_\bot \vec{E}_\bot - \frac{i}{\mu_0 \omega} \left( \vec{k} \times \vec{k} \times \vec{E} \right)_\bot - \tensor{\sigma}_\bot \cdot \vec{E}_\bot \\
  0 &= - i \omega \epsilon_\parallel E_\parallel - \frac{i}{\mu_0 \omega} \left( \vec{k} \times \vec{k} \times \vec{E} \right)_\parallel - \sigma_\parallel E_\parallel
\end{align}

Note that $\sigma_\parallel = \sigma_0 / \left( 1 - i \frac{m \omega \sigma_0}{n e^2} \right)$. 

Note that $\vec{k} \times \vec{k} \times \vec{E} = \vec{k} \vec{k} \cdot \vec{E} - k^2 \vec{E}$. In the electrostatic limit, $\vec{k} \cdot \vec{E} = 0$. 

Say that $\theta$ is the angle between \zhat and $\hat{k}$. Parallel propagation is $\theta = 0$. 

Without loss of generality, we can line up \zhat with the zeroth-order magnetic field, and \xhat with the perpendicular component of the wave vector. Then, rearranging a few constants and combining the equations, we get:

\begin{align}
  0 =& \mmm{ \frac{\epsilon_\bot}{\epsilon_0} }{0}{0}
           {0}{ \frac{\epsilon_\bot}{\epsilon_0} }{0}
           {0}{0}{ \frac{\epsilon_\parallel}{\epsilon_0} } \cdot \vec{E}
       + \frac{c^2 k^2}{\omega^2} 
       \mmm{\sin^2\theta}{0}{\sin \theta \cos \theta}
           {0}{0}{0}
           {\sin \theta \cos \theta}{0}{\cos^2 \theta} \cdot \vec{E} \\
    & - \frac{c^2 k^2}{\omega^2} 
       \mmm{1}{0}{0}
           {0}{1}{0}
           {0}{0}{1} \cdot \vec{E}
       - \frac{i}{\epsilon_0 \omega}
       \mmm{\sigma_P}{- \sigma_H}{0}
           {\sigma_H}{\sigma_P}{0}
           {0}{0}{\sigma_\parallel} \cdot \vec{E}
\end{align}

Maybe instead write this as...

\begin{align}
  \tensor{D} \cdot \vec{E} &= 0
\end{align}

In order to have solutions with nonzero $\vec{E}$, we need:

\begin{align}
  \left| \tensor{D} \right| &= 0
\end{align}

Where

\begin{align}
  \tensor{D} &= 
  \mmm{ \frac{\epsilon_\bot}{\epsilon_0} }{0}{0}
      {0}{ \frac{\epsilon_\bot}{\epsilon_0} }{0}
      {0}{0}{ \frac{\epsilon_\parallel}{\epsilon_0} }
  + \frac{c^2 k^2}{\omega^2} 
  \mmm{\sin^2\theta}{0}{\sin \theta \cos \theta}
      {0}{0}{0}
      {\sin \theta \cos \theta}{0}{\cos^2 \theta}
  - \frac{c^2 k^2}{\omega^2} 
  \mmm{1}{0}{0}
      {0}{1}{0}
      {0}{0}{1}
  - \frac{i}{\epsilon_0 \omega}
  \mmm{\sigma_P}{- \sigma_H}{0}
      {\sigma_H}{\sigma_P}{0}
      {0}{0}{\sigma_\parallel}
\end{align}

% -----------------------------------------------------------------------------
% -----------------------------------------------------------------------------
% -----------------------------------------------------------------------------
\section{Electrostatic Limit}


















