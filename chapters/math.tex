% %%%%%%%%%%%%%%%%%%%%%%%%%%%%%%%%%%%%%%%%%%%%%%%%%%%%%%%%%%%%%%%%%%%%%%%%%%%%%
% %%%%%%%%%%%%%%%%%%%%%%%%%%%%%%%%%%%% Compute the Dispersion Relation for Tuna
% %%%%%%%%%%%%%%%%%%%%%%%%%%%%%%%%%%%%%%%%%%%%%%%%%%%%%%%%%%%%%%%%%%%%%%%%%%%%%

\chapter{Waves in Cold Resistive Plasmas}
  \label{ch_math}

\todo{Sketch out what a dispersion relation is, why it works, why it's interesting. }

\todo{Explain that the inclusion of conductivity is novel. }

\todo{This chapter works out the sorts of waves that might be expected in the numerical model. It starts with the same equations that are used by the model -- Maxwell's equations and \ohmlaw. The resulting dispersion relation is too high-ordered for a direct solution, so several limits of interest are considered. } 

\section{The Dispersion Tensor}

\todo{At this end of this chapter, there will be a discussion of what is specifically interesting about the findings. That doesn't really exist yet. Or... actually, should we discuss interesting things as we find them? Note that we find $\omega^2 = k^2 \va^2$ several times. }

Cold, linearlized \amplaw and \farlaw. The vector $\vec{B}$ is the perturbation to the zeroth-order magnetic field. 
\begin{align}
  \ddt \vec{B} &= -\curl{E} & \tensor{\epsilon} \cdot \ddt \vec{E} &= \oomz \curl{B} - \vec{J}
\end{align}

\ohmlaw. Electron inertial effects are included in the parallel direction. See \cref{ch_inertia}. 
\begin{align}
  \frac{\me}{n e^2} \ddt J_\parallel & = 
    \sz E_\parallel - J_\parallel &
  0 & = 
    \tensor{\sigma}_\bot \cdot \vec{E}_\bot - \vec{J}_\bot
\end{align}

Suppose that the fields and currents are resonating as $\exp \arg{i \vec{k} \cdot \vec{x} - i \omega t }$. Evaluate the derivatives. Eliminate magnetic fields and currents. 
\begin{alignat}{4}
  \label{disp_para}
  & 0 = E_\parallel && + \frac{c^2}{\omega^2} \lr{ \vec{k} \, \vec{k} \cdot \vec{E} - k^2 \vec{E} }_\parallel && + \frac{i \op^2 }{\omega \lr{\nu - i \omega} } && E_\parallel \\
  \label{disp_perp}
  & 0 = \vec{E}_\bot && + \frac{\va^2}{\omega^2} \lr{ \vec{k} \, \vec{k} \cdot \vec{E} - k^2 \vec{E} }_\bot && + \frac{i}{\ep \omega} \tensor{\sigma}_\bot \cdot && \vec{E}_\bot
\end{alignat}

The above expression makes use of the vector identity $\vec{k} \times \vec{k} \times \vec{E} = \vec{k} \, \vec{k} \cdot \vec{E} - k^2 \vec{E}$. The \Alfven speed, speed of light, plasma frequency, and parallel conductivity are defined in the usual way: 
\begin{align}
  \label{def_basics}
  \va^2 & \equiv \frac{1}{\mz \ep} &
  c^2 & \equiv \frac{1}{\mz \ez} &
  \op^2 & \equiv \frac{n e^2}{\me \ez} &
  \sz & \equiv \frac{n e^2}{\me \nu}
\end{align}

Note that this definition of the \Alfven speed takes into account the displacement current correction which is important when \va approaches $c$. 

Without loss of generality, the wave vector $\vec{k}$ can be said to lie in the \x-\z plane; the distinction between the \x and \y directions is revisited in \cref{sec_math_implications}. Let $\theta$ be the angle between $\vec{k}$ and the zeroth-order magnetic field. 

Then \cref{disp_para,disp_perp} can be combined into the usual dispersion tensor form, $\dispersiontensor \cdot \vec{E} = 0$, where 
\begin{align}
  \dispersiontensor & = \tensor{ \mathbb{I} }
                      + \frac{k^2}{\omega^2} 
                        \mmm{-\va^2 \cos^2 \theta}{0}{\va^2 \sin \theta \cos \theta}
                            {0}{-\va^2}{0}
                            {c^2 \sin \theta \cos \theta}{0}{-c^2 \sin^2 \theta}
                      + \frac{i}{\omega}
                        \mmm{ \frac{\sp}{\ep} }{-\frac{\sh}{\ep} }{0}
                            { \frac{\sh}{\ep} }{ \frac{\sp}{\ep} }{0}
                            {0}{0}{ \frac{\op^2}{\nu - i \omega} }
\end{align}

Nontrivial solutions exist only when $\left| \dispersiontensor \right| = 0$. This gives rise to a seventh-order polynomial in $\omega$, so it's necessary to consider limits of particular interest. 

% =============================================================================
% =============================================================================
% =============================================================================
\section{Parallel Propagation Limit}

Parallel propagation is a naive representation of a field line resonance; the wave vector moves energy along the field line. Taking $\theta=0$, $\dispersiontensor$ simplifies to
\begin{align}
  \label{parallel_dispersion_tensor}
  \dispersiontensor_\parallel & = \tensor{ \mathbb{I} }
                      - \frac{k^2 \va^2}{\omega^2} 
                        \mmm{1}{0}{0}
                            {0}{1}{0}
                            {0}{0}{0}
                      + \frac{i}{\omega}
                        \mmm{ \frac{\sp}{\ep} }{-\frac{\sh}{\ep} }{0}
                            { \frac{\sh}{\ep} }{ \frac{\sp}{\ep} }{0}
                            {0}{0}{ \frac{\op^2}{\nu - i \omega} }
\end{align}

Conveniently, parallel and perpendicular polarizations are not coupled in \cref{parallel_dispersion_tensor}. 

% -----------------------------------------------------------------------------
% -----------------------------------------------------------------------------
% -----------------------------------------------------------------------------
\subsection{Parallel Polarization}

The parallel component of the determinant of \cref{parallel_dispersion_tensor} gives
\begin{align}
  \omega^2 + i \nu \omega - \op^2 = 0
\end{align}

With no $k$ dependence this expression doesn't describe a wave per se, so much as a resonant frequency. Solving directly, 
\begin{align}
  \omega & = -\frac{i \nu}{2} \pm \sqrt{ \op^2 - \frac{\nu^2}{4} }
\end{align}

The plasma frequency significantly exceeds the collision frequency everywhere except a narrow strip at the ionospheric boundary of the model. Expanding in large $\op$ gives
\begin{align}
  \omega^2 & = \op^2 - i \nu \op + ...
\end{align}

\todo{Actually, double-check this. How does $\nu$ compare to $\op$ at the ionospheric boundary when there is no Boris factor? }

This is the plasma oscillation. 

% -----------------------------------------------------------------------------
% -----------------------------------------------------------------------------
% -----------------------------------------------------------------------------
\subsection{Perpendicular Polarization}

The perpendicular components of the determinant of \cref{parallel_dispersion_tensor} give
\begin{align}
  \omega^4 + 2 i \frac{\sp}{\ep} \omega^3
  - \lr{ 2 k^2 \va^2 + \frac{\sp^2 + \sh^2}{\ep^2} } \omega^2
  - 2 i k^2 \va^2 \frac{\sp}{\ep} \omega
  + k^4 \va^4 & = 0
\end{align}

This can be solved in exact form. Noting that $\pm$ and $\opm$ are independent,
\begin{align}
  \omega = \lr{ \frac{\pm \sh - i \sp}{2 \ep} } \opm \sqrt{ k^2 \va^2 + \lr{ \frac{\pm \sh - i \sp}{2 \ep} }^2 }
\end{align}

Over the vast majority of a field line, $k \va \gg \frac{\sp}{\ep}$ and $k \va \gg \frac{\sh}{\ep}$. 
\begin{align}
  \omega^2 = k^2 \va^2 \opm k \va \frac{\sh \pm i \sp}{\ep} + ...
\end{align}

This is the \Alfven wave, evidently split by the ionospheric conductivity, propagating along the zeroth-order magnetic field line. 

% =============================================================================
% =============================================================================
% =============================================================================
\section{Perpendicular Propagation Limit}

A wave's ability to propagate across field lines is also of interest. When $\theta = \frac{\pi}{2}$, $\dispersiontensor$ simplifies to
\begin{align}
  \label{perp_dispersion_tensor}
  \dispersiontensor_\bot & = \tensor{ \mathbb{I} }
                      - \frac{k^2}{\omega^2} 
                        \mmm{0}{0}{0}
                            {0}{\va^2}{0}
                            {0}{0}{c^2}
                      + \frac{i}{\omega}
                        \mmm{ \frac{\sp}{\ep} }{-\frac{\sh}{\ep} }{0}
                            { \frac{\sh}{\ep} }{ \frac{\sp}{\ep} }{0}
                            {0}{0}{ \frac{\op^2}{\nu - i \omega} }
\end{align}

As in the parallel propagation case, the parallel and perpendicular components of the determinant are decoupled. 

% -----------------------------------------------------------------------------
% -----------------------------------------------------------------------------
% -----------------------------------------------------------------------------
\subsection{Parallel Polarization}

The parallel component of the determinant of \cref{perp_dispersion_tensor} gives
\begin{align}
  \omega^3 + i \nu \omega^2
  - \lr{ k^2 c^2 + \op^2 } \omega
  - i k^2 c^2 \nu & = 0
\end{align}

The above expression can be solved exactly, but the resulting expressions are too long to be useful. 

Expanding the solution in large $\op$ gives the O mode: a compressional wave with a parallel electric field. 
\begin{align}
  \omega^2 & = k^2 c^2 + \op^2 - i \nu \op + ...
\end{align}


% -----------------------------------------------------------------------------
% -----------------------------------------------------------------------------
% -----------------------------------------------------------------------------
\subsection{Perpendicular Polarization}

The perpendicular components of the determinant of \cref{perp_dispersion_tensor} give
\begin{align}
  \omega^3 + 2 i \frac{\sp}{\ep} \omega^2
  - \lr{ 2 k^2 \va^2 + \frac{\sp^2 + \sh^2}{\ep^2} } \omega
   - i k^2 \va^2 \frac{\sp}{\ep} & = 0
\end{align}

Again, the roots of the cubic are impractically long. 

Expanding in large conductivity, as is expected near the ionospheric boundary, gives
\begin{align}
  \omega^2 & = k^2 \va^2 + \lr{ \frac{\sh \pm i \sp}{\ep} }^2 + ...
\intertext{Whereas expanding in small conductivity, as is expected far from the boundary, gives}
  \omega^2 & = k^2 \va^2 \pm i k \va \frac{\sp}{\ep} + ...
\end{align}

Both of which are \Alfven waves. 

% =============================================================================
% =============================================================================
% =============================================================================
\section{High Altitude Limit}

At high altitude, where the density is very low compared to the ionosphere, it's reasonable to approximate $\sp \rightarrow 0$ and $\sh \rightarrow 0$ and $\nu \rightarrow 0$. 

In this case, $\dispersiontensor$ simplifies to

\begin{align}
  \label{high_dispersion_tensor}
  \dispersiontensor_{\infty} & = \tensor{ \mathbb{I} }
                      - \frac{k^2}{\omega^2} 
                        \mmm{\va^2 \cos^2 \theta}{0}{-\va^2 \sin \theta \cos \theta}
                            {0}{\va^2}{0}
                            {-c^2 \sin \theta \cos \theta}{0}{c^2 \sin^2 \theta}
                      - \frac{\op^2}{\omega^2}
                        \mmm{0}{0}{0}
                            {0}{0}{0}
                            {0}{0}{1}
\end{align}

In this case it's the azimuthal polarization that decouples from the other two. 

% -----------------------------------------------------------------------------
% -----------------------------------------------------------------------------
% -----------------------------------------------------------------------------
\subsection{Azimuthal Polarization}

The azimuthal component of the determinant of \cref{high_dispersion_tensor} simply gives the \Alfven wave. 
\begin{align}
  \omega^2 & = k^2 \va^2
\end{align}

% -----------------------------------------------------------------------------
% -----------------------------------------------------------------------------
% -----------------------------------------------------------------------------
\subsection{Meridional Polarization}

The components of the determinant of \cref{high_dispersion_tensor} which fall in the meridional plane give, taking $k_\bot \equiv k \sin \theta$ and $k_\parallel \equiv k \cos \theta$,
\begin{align}
  \omega^4 
  - \lr{ k_\parallel^2 \va^2 + k_\bot^2 c^2 + \op^2 } \omega^2
  + k_\parallel^2 \va^2 \op^2  & = 0
\end{align}

The above expression is quadratic in $\omega^2$, and thus can be solved directly. 
\begin{align}
  \omega^2 & = \frac{1}{2} \lr{ k_\parallel^2 \va^2 + k_\bot^2 c^2 + \op^2 }
  \pm \sqrt{ \frac{1}{4} \lr{ k_\parallel^2 \va^2 + k_\bot^2 c^2 + \op^2 }^2 
    - k_\parallel^2 \va^2 \op^2 }
\end{align}

Noting that $\op$ is very large, the two roots simplify to
\begin{align}
  \omega^2 & = k_\parallel^2 \va^2 + ... \\
  \omega^2 & = k_\parallel^2 \va^2 + k_\bot^2 c^2 + \op^2 + ...
\end{align}

% =============================================================================
% =============================================================================
% =============================================================================
\section{Implications for This Work}
  \label{sec_math_implications}

% -----------------------------------------------------------------------------
% -----------------------------------------------------------------------------
% -----------------------------------------------------------------------------
\subsection{Mode Coupling}

\todo{Have we got enough/appropriate math here to talk about how $\sh$ rotates fields at the ionosphere? Doubtful. }

% -----------------------------------------------------------------------------
% -----------------------------------------------------------------------------
% -----------------------------------------------------------------------------
\subsection{High Modenumber Cutoff}

As $\azm$ becomes large, it puts a lower bound on the modenumber, and thus on the frequency. For an \Alfven wave, $\omega^2 = k^2 \va^2$,
\begin{align}
  k \ge k_\phi = \frac{\azm}{2 \pi r} \qquad \text{so} \qquad \omega \ge \frac{\azm}{2 \pi r} \va
\end{align}

Any wave with a frequency below this threshold will become evanescent. 

\todo{This point is pretty important. It's revisited in \cref{sec_driving}. }


