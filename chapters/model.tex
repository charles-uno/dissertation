% %%%%%%%%%%%%%%%%%%%%%%%%%%%%%%%%%%%%%%%%%%%%%%%%%%%%%%%%%%%%%%%%%%%%%%%%%%%%%
% %%%%%%%%%%%%%%%%%%%%%%%%%%%%%%%%%%%%%%%%%%%%%% Description of Numerical Model
% %%%%%%%%%%%%%%%%%%%%%%%%%%%%%%%%%%%%%%%%%%%%%%%%%%%%%%%%%%%%%%%%%%%%%%%%%%%%%

\chapter{Numerical Model}
\label{model_chapter}

It's called Tuna! 

% -----------------------------------------------------------------------------
% -----------------------------------------------------------------------------
% -----------------------------------------------------------------------------
\section{Coordinates and Notation}

\subsection{Covariance and Contravariance}

This probably goes in the differential geometry appendix?

\subsection{Nonorthogonal Dipole Coordinates}

2.5 dimensions. It doesn't have to mean periodicity. We are using it, instead, to look at an azimuthally-narrow phenomenon. 

% -----------------------------------------------------------------------------
% -----------------------------------------------------------------------------
% -----------------------------------------------------------------------------
\section{Maxwell's Equations and Ohm's Law}

% -----------------------------------------------------------------------------
% -----------------------------------------------------------------------------
% -----------------------------------------------------------------------------
\section{Driving}

Compressional driving can't penetrate at high $m$. 

Pc4 constrains location and frequency. 

Ring current determines frequency and power. 

% -----------------------------------------------------------------------------
% -----------------------------------------------------------------------------
% -----------------------------------------------------------------------------
\section{Boundary Conditions}

% -----------------------------------------------------------------------------
% -----------------------------------------------------------------------------
% -----------------------------------------------------------------------------
\section{Coupling to the Atmosphere}

Laplace's Equation. 

Comparison to spherical harmonics. 

Ionospheric jump condition. 

% -----------------------------------------------------------------------------
% -----------------------------------------------------------------------------
% -----------------------------------------------------------------------------
\section{Parallel Electric Fields}

Or, field aligned currents. 

Boris approximation for stability. LFM, BATRUS use this? 


% -----------------------------------------------------------------------------
% -----------------------------------------------------------------------------
% -----------------------------------------------------------------------------
\section{Discussion of Model Parameters}

Ring current power spectrum. Find more storms to look at! 



Some drive parameters don't really matter. Exact drive position, for example, in r and $\theta$? Boundary conditions (at least for the phenomena we care about)?



% -----------------------------------------------------------------------------
% -----------------------------------------------------------------------------
% -----------------------------------------------------------------------------
\section{Optimization of Model}

Extensive precomputation of coefficients. 

Reduced memory demand? Not clear if this is still true when we're tracking C and F. 

OpenMP. 

MPI benchmarks. C++ vs Fortran. Can we get some plots for this? Even just 2 ranks, neglecting the atmosphere. 













