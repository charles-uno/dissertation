% %%%%%%%%%%%%%%%%%%%%%%%%%%%%%%%%%%%%%%%%%%%%%%%%%%%%%%%%%%%%%%%%%%%%%%%%%%%%%
% %%%%%%%%%%%%%%%%%%%%%%%%%%%%%%%%%%%%%%%%%%%%%% Introduction to Pc4 Pulsations
% %%%%%%%%%%%%%%%%%%%%%%%%%%%%%%%%%%%%%%%%%%%%%%%%%%%%%%%%%%%%%%%%%%%%%%%%%%%%%

\chapter{Magnetic Pulsations}
\label{ch_pc4s}

\todo{This chapter will lay out the basics of ULF waves, specifically Pc4 pulsations and giant pulsations. Notable steps in theory and observation will be summarized. For now, it's mostly a disorganized mishmash of citations. }

%
% From W J Hughes' Magnetospheric ULF Waves: A Tutorial With a Historical Perspective
%

% rolf 1920 for early observation of Pi2
% angenheister 1931 for early giant pulsation
% Dungey 1954, 1963 for geomagnetic pulsations along field lines. 

% Jacobs et al 1964 has Pi and Pc classifications
% Sigura 1961 showed simultaneous observations at north and south foot points
% nagata et al 1963 showed that they were standing waves

% Patel 1965 made observations in space, and matched them to observations on the ground
% Cummings et al 1969 numerically integrated Dungey's equations to estimate eigenfrequencies
% soviets discovered that different pulsations have different sources in the 1970s; see troitskaya 1993 and greenstadt and russell 1993
% troitskaya 1969 showed that Pc3 can sometimes be explained by a sudden change in the size of the magnetosphere following an impulse. an example -- this must have also been seen earlier. because... 
% Bolshakova and Troitskaya 1968 showed that Pc3 observation depends on IMF N/S. Iff IMF is within about 50 degrees of the earth-sun linem Pc3 is observable on the ground. 
% Troitskaya et al 1971 showed Pc3 frequencies at L=3 depend on IMF magnitude
% Gul'elmi 1974 and Kovner et al gave theoretical justification for these observations. 
% Russell and Hoppe 1981 made observations upstream of the bow shock to unify some ideas. upstream waves are driven by ion cyclotron instability, which depends on IMF magnitude. 

% Gringauz et al 1970 found that solar wind number density affects Pc3 and Pc4 periods (oppositely). The Pc4 effect is a standing question. 

% Samson et al 1971 new observations with digital recording and arrays near the plasmapause! they showed MLT dependence of peak amplitude, and different regions of opposite circular polarization. 
% Dungey 1954 had suggested KH as a wave energy source, which explained the flipping circular polarization at noon. 
% Southwood 1974 went back to the equations and came up with a reason for resonance. waves hitting the bow shock propagate in as an evanescent fast mode. when it gets to a resonant field line, the fast mode couples to the shear mode. the energy tunnels. this only happens if dissipation is finite. 
% Chen and Hasegawa 1974 did too, independently
% Newton et al 1978 showed that joule dissipation at the ionosphere is dominant, and that the amount of dissipation determines the width of the resonance. 

% kivelson et al 1984 argued for cavity mode eigenfrequencies. Pc5. 
% kivelson and southwood 1985, 1986 did analytical work to defend it. 
% Lee and Lysak 1991 did numerical work in a dipole geometry. 

% Harrold and Samson 1992 The waves probably have to be reflected by the bow shock, not by the magnetopause, to line up with the observed eigenfrequencies. 1 to 3 mHz. 

``To summarize, the general buffetting of the magnetosphere by variations in the solar wind dynamic pressure, or perhaps by sporadic magnetic reconnection, provides a broad band energy source to the magnetosphere. The magnetospheric cavity as a whole rings at its own eigenfrequencies, thus transporting energy at just those frequencies to field lines deep in the magnetosphere. Those field lines whose eigenfrequencies match one of the cavity eigenfrequencies couple to the cavity mode and resonate strongly, producing the classical field line resonance signature.\cite{hughes_1994}'' 

% allan et al 1985 did early observation of ULF waves (Pc5 and Pc4) from waveform plots. 
% Waters et al 1991 ground-based observation of Pc3 and Pc4 before spacecraft existed at small L. 
% Menk et al 1993 also

Dungey's \Alfven wave paper\cite{dungey_1954}. 

High-modenumber ULF pulsations are damped by the ionosphere, making it more difficult to observe them on the ground\cite{hughes_1976}. Small structures are also damped; resonances narrower than $\sim \SI{100}{km}$ aren't visible on the ground. 

Multi-spacecraft observation of a ULF wave with very high modenumber (70+) and no apparent ground signature\cite{takahashi_2013}. 

A recent survey of Van Allen Probe data showed that Pc4 pulsations -- the poloidal ones, at least -- occur primarily during geomagnetically active times, near the plasmapause, over just a handful of hours of dayside MLT\cite{dai_2015}. This confirmed and refined older work\cite{engebretson_1987}. 

ULF waves have been shown to correlate with pulsating aurora and with chorus\cite{jaynes_2015}. It's believed that (in the case presented) substorm injection drove Pc4-5 pulsations, which modulated chorus waves, which pitch-angle scattered electrons with energies on the order of \SI{10}{\kilo\eV}. 

Energetic ring current and radiation belt particles can be accelerated by ULF waves through drift or drift-bounce resonance\cite{southwood_1976}. 

A Hall-conducting ionosphere reflects ULF waves\cite{hughes_1974}. 

``It is not clear why noncompressional [high-\azm] Pc4 poloidal waves, which are presumably driven by instability within the magnetosphere, preferrentially occur on the dayside.\cite{dai_2015}''

Poloidal and toroidal ULF polarizations are treated differently by the ionosphere\cite{greifinger_1968} (or, more recently, \cite{fujita_1988}).  

\todo{Theoretical consideration of decay vs propagation, by frequency. Lysak and Yoshikawa 2006. }

\todo{Fishbone instability. McGuire 1983, Chen 1984. Similar phenomenon, but for lab plasmas. }

An ideal poloidal mode decays to the toroidal mode in the presence of curved magnetic field lines\cite{radoski_1974} or a gradient in the \Alfven speed\cite{mann_1995}. The time is proportional to the azimuthal wavenumber\cite{mann_1995}. An analytical follow-up agreed with the numerical work\cite{mann_1997}. 

\todo{Mann gives the poloidal-to-toroidal decay time to be $\tau = \frac{d \lambda}{d \omega_A'}$, where $\lambda = \frac{\azm}{2 \pi r}$ and $\omega_A'$ is the spatial derivative of the \Alfven bounce frequency, but this doesn't seem to line up. When $\tau$ is computed using our \Alfven bounce frequencies, the result is much less than \SI{1}{\second}. Double-checking is necessary. }





