% %%%%%%%%%%%%%%%%%%%%%%%%%%%%%%%%%%%%%%%%%%%%%%%%%%%%%%%%%%%%%%%%%%%%%%%%%%%%%
% %%%%%%%%%%%%%%%%%%%%%%%%%%%%%%%%%%%%%%%%%%%%% Simulations of Giant Pulsations
% %%%%%%%%%%%%%%%%%%%%%%%%%%%%%%%%%%%%%%%%%%%%%%%%%%%%%%%%%%%%%%%%%%%%%%%%%%%%%

\chapter{Relevance to Giant Pulsations}
  \label{ch_pgs}

Giant pulsations are (probably\cite{takahashi_2011}) fundamental mode poloidal Pc4 pulsations with frequencies around \SI{10}{\mHz} and azimuthal modenumbers around \num{20}. They are large, and can sometimes be observed on the ground. 

This model -- with its so-far sinusoudal driving -- isn't able to comment on the fact that giant pulsations tend to be near-monochromatic. 

However, just about everything else about giant pulsations lines up with what the model shows... in fact, this might not even be its own chapter. It might be a section of \cref{ch_lifetimes}. 

Giant pulsations tend to have modenumbers around 20. This is exactly what we would expect. Very low modenumbers rotate from the poloidal mode to the toroidal mode very quickly. Even with ongoing driving and high conductivity, no satisfying poloidal resonance could be produced with an azimuthal modenumber of order unity. 

Giant pulsations are not observed with very high modenumbers, on the other hand, because high-\azm waves are screened out by the ionosphere. This was shown theoretically\cite{hughes_1976} in the seventies. Recent observations indicate the same; in 2013, Takahashi\cite{takahashi_2013} had a multispacecraft observation of a poloidal Pc4 with $\azm \gtrsim 70$. No ground signature. This model gives consistent results. 

\todo{Show some ground signatures. They're weak for high-\azm runs. }

Giant pulsations often drift azimuthally. This model can't resolve azimuthal drift directly, of course, but can fake it by looking at complex phase. 

I have seen some indication of complex phase rotation in ground magnetic fields, indicating horizontal motion on the order of $\SI{e-1}{\degree/\second}$. But it's not clear if it's real. For the bulk of the simulation, poloidal fields are completely real and toroidal fields are completely imaginary. That gets a bit muddled at the ionosphere due to the Hall conductivity. I haven't quite figured out where the complex phase of the ground signatures fits into that picture. 





