% %%%%%%%%%%%%%%%%%%%%%%%%%%%%%%%%%%%%%%%%%%%%%%%%%%%%%%%%%%%%%%%%%%%%%%%%%%%%%
% %%%%%%%%%%%%%%%%%%%%%%%%%%%%%%%%%%%%%%%%%%%%% Validate the Model Against RBSP
% %%%%%%%%%%%%%%%%%%%%%%%%%%%%%%%%%%%%%%%%%%%%%%%%%%%%%%%%%%%%%%%%%%%%%%%%%%%%%

\chapter{Comparison to Van Allen Probes}
  \label{ch_rbsp}

\todo{Do we need to acknowledge that a moving spacecraft might have these fields shifted per $\vec{E} + \cross{U}{B}=0$? }

\todo{I have met a handful of times with John (Wygant) to discuss how to get some RBSP data into this dissertation.}

\todo{John has recently become interested in the long (\SI{26}{\day} or so) cycle in the size of the plasmapause.}

\todo{Lei recently compiled a list of several hundred Pc4 events seen by RBSP\cite{dai_2015}. He binned them by plasmapause location, noting that there were a significant number of events in the entire interval $4 \lesssim L \lesssim 6$. However, he was just writing about occurrence rate, so he didn't delve into how the large-plasmasphere Pc4s were different from the small-plasmasphere ones.}

\todo{I'll bet I can find a pattern or two in Pc4 behavior based on plasmapause location. It'll be interesting to see if the model exhibits the same pattern as the plasmapause location is adjusted.}

\todo{I emailed Lei about this earlier this week. He sent over his event list.}

\todo{Sheng has agreed to show me how to access RBSP data.}

\todo{There is some concern about the rules attached to funding. Apparently NASA fundees aren't allowed to be involved in two-person collaborations with international scientists, or something? We'll have to get that figured out. }

% =============================================================================
% =============================================================================
% =============================================================================
\section{Event Selection}

% -----------------------------------------------------------------------------
% -----------------------------------------------------------------------------
% -----------------------------------------------------------------------------
\subsection{Identifying the Fundamental Harmonic}

Takahashi\cite{takahashi_2013} spells this out explicitly. Fundamental mode, south of the equator, V should lead B by 90 degrees. 

Per Takahashi\cite{takahashi_2011}, phase lag can be used to distinguish the fundamental harmonic from the second harmonic. 

Chisham and Orr\cite{chisham_1991} argue that around \SI{7}{\RE}, frequency around \SI{10}{\mHz} precludes higher harmonics. Or maybe look at \cite{green_1985}?

Green\cite{green_1979} and Cummings\cite{cummings_1969} talk about the frequencies for ideal toroidal modes. 

Dai\cite{dai_2015} says to look at \cite{takahashi_2011} and \cite{dai_2013} for unambiguous identification of the fundamental mode. 

% =============================================================================
% =============================================================================
% =============================================================================
\section{Something Something Results}

% -----------------------------------------------------------------------------
% -----------------------------------------------------------------------------
% -----------------------------------------------------------------------------
\subsection{Something Something Results}









