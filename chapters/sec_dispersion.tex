% =============================================================================
% =============================================================================
% =============================================================================
\section{Dispersion Relation}
  \label{sec_math}

Before delving into the implementation of the numerical model, it's instructive to consider the fundamental equations of waves in a cold, resistive plasma. 

The evolution of electric and magnetic fields is governed by \amplaw and \farlaw. Notably, the present work takes the linearized versions of Maxwell's equations; the vectors \vec{E} and \vec{B} indicate perturbations above the zeroth-order magnetic field. 
\begin{align}
  \label{math_maxwells}
  \tensor{\epsilon} \cdot \ddt \vec{E} &= \oomz \curl{B} - \vec{J} & \ddt \vec{B} &= -\curl{E}
\end{align}

Current follows \ohmlaw. For currents perpendicular to the background magnetic field, Kirchhoff's formulation is sufficient. Conductivity is much higher along the magnetic field lines\footnote{The dipole coordinate system is defined rigorously in \cref{sec_coords}; at present, it's sufficient to take \zhat parallel to the zeroth-order magnetic field, and \xhat and \yhat perpendicular to \zhat (and to each other). }, so it's appropriate to also include the electron inertial term\footnote{Electron inertial effects are not included in the model described in \cref{sec_tuna}, or in the results in \cref{ch_results}; however, their implementation and impact are explored in \cref{sec_inertia}. }. 
\begin{align}
  \label{math_ohms}
  0 &= \tensor{\sigma}_\bot \cdot \vec{E}_\bot - \vec{J}_\bot &
  \frac{\me}{n e^2} \ddt J_z &= \sz E_z - J_z
\end{align}

The derivatives in \cref{math_maxwells,math_ohms} can be evaluated by performing a Fourier transform. They can then be combined, eliminating the magnetic field and current terms. 
\begin{align}
  \label{disp_setup}
\begin{alignedat}{4}
  & 0 = \vec{E}_\bot && + \frac{\va^2}{\omega^2} \lr{ \vec{k} \times \vec{k} \times \vec{E} }_\bot && + \frac{i}{\ep \omega} \tensor{\sigma}_\bot \cdot && \vec{E}_\bot \\
  & 0 = E_z && + \frac{c^2}{\omega^2} \lr{ \vec{k} \times \vec{k} \times \vec{E} }_z && + \frac{i \op^2 }{\omega \lr{\nu - i \omega} } && E_z
\end{alignedat}
\end{align}

The speed of light, \Alfven speed (including the displacement current correction), plasma frequency, and parallel conductivity are defined in the usual way: 
\begin{align}
  \label{def_basics}
  c^2 & \equiv \frac{1}{\mz \ez} &
  \va^2 & \equiv \frac{1}{\mz \ep} &
  \op^2 & \equiv \frac{n e^2}{\me \ez} &
  \sz & \equiv \frac{n e^2}{\me \nu}
\end{align}

Using the vector identity $\vec{k} \times \vec{k} \times \vec{E} = \vec{k} \, \vec{k} \cdot \vec{E} - k^2 \vec{E}$, \cref{disp_setup} can be reassembled into a single expression, 
\begin{align}
  \label{disp}
  0 &= \lr{ \tensor{ \mathbb{I} } + \frac{1}{\omega^2} \tensor{V}^2 \cdot \vec{k} \, \vec{k} - \frac{k^2}{\omega^2} \tensor{V}^2 + \frac{i}{\omega} \tensor{\Omega} } \cdot \vec{E}
\end{align}

Where $\tensor{ \mathbb{I} }$ is the identity tensor and in \x-\y-\z coordinates\footnote{Note that the present definition of \tensor{\Omega} differs slightly from that used in \cref{sec_eqns}. }, 
\begin{align}
  \tensor{V}^2 &\equiv 
    \mmm{\va^2}{0}{0}
        {0}{\va^2}{0}
        {0}{0}{c^2} &
  & \text{and} &
  \tensor{\Omega} &\equiv 
    \mmm{ \frac{\sp}{\ep} }{ \frac{-\sh}{\ep} }{0}
        { \frac{\sh}{\ep} }{ \frac{\sp}{\ep} }{0}
        {0}{0}{ \frac{\op^2}{\lr{\nu - i \omega}} } 
\end{align}

In \cref{disp}, the expression in parentheses is the dispersion tensor. Nontrivial solutions exist only when its determinant is zero. This gives rise to a seventh-order polynomial in $\omega$, so rather than a direct solution it's necessary to consider limits of specific interest. 




... \\ \\ \\






% -----------------------------------------------------------------------------
% -----------------------------------------------------------------------------
% -----------------------------------------------------------------------------
\subsection{Parallel or Perpendicular Propagation}
  \label{sec_par_perp}

\todo{Parallel propagation is ineresting because it's a naive representation of a field line resonance. Perpendicular propagation is interesting because the movement of energy across field lines is a topic of particular interest. In both cases, the cross terms in $\vec{k} \, \vec{k}$ vanish, decoupling the parallel and perpendicular polarizations. }

\todo{Without loss of generality, the wave vector $\vec{k}$ is presumed to lie in the \x-\z plane. The distinction between the azimuthal and crosswise directions is revisited later. }

\todo{ACTUALLY: does this even matter? If we keep $\phi$ in there in addition to $\theta$, does it cancel out? Seems like it should. }

%\subsection{Parallel Polarization}

In the parallel and perpendicular propagation limits respectively, the parallel factors of the determinants of \dispersiontensor are:
\begin{align}
  \label{parallel_math_setup}
  \begin{split}
  % Parallel propagation, parallel polarization. 
  0 &= \omega^2 + i \nu \omega - \op^2 \\
  % Perpendicular propagation, parallel polarization. 
  0 &= \omega^3 + i \nu \omega^2
  - \lr{ k^2 c^2 + \op^2 } \omega
  - i k^2 c^2 \nu
  \end{split}
\end{align}

Both expressions in \cref{parallel_math_setup} can be solved directly, though the solution to the (cubic) perpendicular propagation case is too long to be useful. After expanding with respect to a very large plasma frequency, the solutions can be written (for parallel and perpendicular propagation respectively): 
\begin{align}
  \label{parallel_math_final}
  \begin{split}
  % Parallel propagation, parallel polarization. 
  \omega^2 & = \op^2 - i \nu \op + ... \\
  % Perpendicular propagation, parallel polarization. 
  \omega^2 & = k^2 c^2 + \op^2 - i \nu \op + ...
  \end{split}
\end{align}

The first expression in \cref{parallel_math_final} -- describing the parallel-polarized component of a parallel-propagating wave -- doesn't describe a wave at all; there's no dependence on the wave vector $k$. Instead, it describes the plasma oscillation. The second expression is the O mode: a compressional wave, oscillating just above the plasma frequency, with a parallel electric field. 

The perpendicular factors of the dispersion tensor's determinant, in the cases of parallel and perpendicular propagation respectively, are: 
\begin{align}
  \label{perpendicular_math_setup}
  \begin{split}
  % Parallel propagation, perpendicular polarization. 
  0 &= \omega^4 + 2 i \frac{\sp}{\ep} \omega^3
  - \lr{ 2 k^2 \va^2 + \frac{\sp^2 + \sh^2}{\ep^2} } \omega^2
  - 2 i k^2 \va^2 \frac{\sp}{\ep} \omega
  + k^4 \va^4 \\
  % Perpendicular propagation, perpendicular polarization. 
  0 &= \omega^3 + 2 i \frac{\sp}{\ep} \omega^2
  - \lr{ 2 k^2 \va^2 + \frac{\sp^2 + \sh^2}{\ep^2} } \omega
   - i k^2 \va^2 \frac{\sp}{\ep}
  \end{split}
\end{align}

First expression of \cref{perpendicular_math_setup} can be solved directly. Noting that $\pm$ and $\opm$ are independent,
\begin{align}
  % Parallel propagation, perpendicular polarization. 
  \omega = \lr{ \frac{\pm \sh - i \sp}{2 \ep} } \opm \sqrt{ k^2 \va^2 + \lr{ \frac{\pm \sh - i \sp}{2 \ep} }^2 }
\end{align}

This wave is propagating along the field line, so the Pedersen and Hall conductivities are small for the vast majority of its trajectory. Expanding, then, gives
\begin{align}
  \label{perp_final_z}
  % Parallel propagation, perpendicular polarization. 
  \omega^2 = k^2 \va^2 \opm k \va \frac{\sh \pm i \sp}{\ep} + ...
\end{align}

The second expression of \cref{perpendicular_math_setup} can also be solved directly, though its roots are impractically long. They are expanded with respect to small conductivities (as would be seen by a perpendicular-propagating wave at high altitude) and at large conductivity (representing the perpendicular-propagating wave within the ionosphere). Respectively, 
\begin{align}
  \label{perp_final_x}
  \begin{split}
  % Perpendicular propagation, perpendicular polarization. Small conductivity. 
  \omega^2 & = k^2 \va^2 \pm i k \va \frac{\sp}{\ep} + ... \\
  % Perpendicular propagation, perpendicular polarization. Large conductivity. 
  \omega^2 & = k^2 \va^2 + \lr{ \frac{\sh \pm i \sp}{\ep} }^2 + ...
  \end{split}
\end{align}

\cref{perp_final_z,perp_final_x} describe \Alfven waves: waves traveling at the \Alfven speed (shifted by the ionospheric conductivity) with electric fields perturbations perpendicular to the zeroth-order magnetic field.  

% -----------------------------------------------------------------------------
% -----------------------------------------------------------------------------
% -----------------------------------------------------------------------------
\subsection{High Altitude Limit}
  \label{sec_high_alt}

In the limit of large radial distance, where the density is very low, it's reasonable to neglect the Pedersen conductivity, the Hall conductivity, and the collision rate. 

Without loss of generality, the wave vector $\vec{k}$ can be presumed to lie in the \x-\z plane\footnote{\todo{The azimuthal-propagation case is discussed in XXX.}}. 

Whereas in \cref{sec_par_perp} the parallel component of the determinant of the dispersion tensor could be factored out, the high altitude limit decouples the azimuthal component from those in the meridional plane. The azimuthal component gives, simply, 
\begin{align}
  \omega^2 & = k^2 \va^2
\end{align}

This solution is analogous to the results in \cref{sec_par_perp}: a wave propagating at the \Alfven speed, with electric field perturbation perpendicular to both the zeroth-order magnetic field and the wave vector. 

The meridional components of the determinant of the high-altitude limit of the dispersion tensor give: 
\begin{align}
  \label{high_meridional_setup}
  0 &= \omega^4 
  - \lr{ k_\parallel^2 \va^2 + k_\bot^2 c^2 + \op^2 } \omega^2
  + k_\parallel^2 \va^2 \op^2
\end{align}

Where $k_\parallel$ and $k_\bot$ are the parallel and crosswise components of the wave vector. The a

\cref{high_meridional_setup} is quadratic in $\omega^2$. Its solution is
\begin{align}
  \omega^2 & = \frac{1}{2} \lr{ k_\parallel^2 \va^2 + k_\bot^2 c^2 + \op^2 }
  \pm \sqrt{ \frac{1}{4} \lr{ k_\parallel^2 \va^2 + k_\bot^2 c^2 + \op^2 }^2 
    - k_\parallel^2 \va^2 \op^2 }
\end{align}

Noting that $\op$ is very large, the two roots simplify to
\begin{align}
  \label{high_meridional_final}
  \begin{split}
  \omega^2 & = k_\parallel^2 \va^2 + ... \\
  \omega^2 & = k_\parallel^2 \va^2 + k_\bot^2 c^2 + \op^2 + ...
  \end{split}
\end{align}

\todo{The first part looks like an \Alfven wave. The second is faster than the plasma frequency, so we don't really care about it. }

% -----------------------------------------------------------------------------
% -----------------------------------------------------------------------------
% -----------------------------------------------------------------------------
\subsection{Something Something Implications}
  \label{sec_implications}

\todo{High \azm limit. }

\todo{Hall conductivity? }

\todo{Waves moving at the \Alfven speed constrain time step. }



The electron inertial length $\frac{c}{\op}$ is on the order of \SI{1}{\km}, smaller than the wavelength of a field line resonance by three or four orders of magnitude. That is, at high altitude, the parallel electric field is expected to be smaller than the perpendicular electric field by a factor of \num{e7} -- perhaps more, depending on how closely the wave vector is aligned to the magnetic field. That seems fine -- note that \cref{fig_electric_field_snapshots} shows that max $E_\parallel$ is 4 to 5 orders larger than max $E_\bot$... plus high altitude is the parallel field's minimum and the perpendicular field's maximum. 



With a bit of algebra, the meridional components of the dispersion tensor from \cref{sec_high_alt} provide a comparison of the parallel and perpendicular electric field magnitudes.
\begin{align}
  \frac{E_\parallel}{E_\bot} &= \frac{- k_\parallel k_\bot c^2 }{ \omega^2 - k_\bot^2 c^2 - \op^2 } \sim \frac{k^2 c^2 }{\op^2}
\end{align}









