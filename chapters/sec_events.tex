
% =============================================================================
% =============================================================================
% =============================================================================
\section{Event Selection}




Note that, in gathering his 860 events, Dai\cite{dai_2015} did not take the $\vec{E} \cdot \vec{B} = 0$ assumption to clean up the electric field data. The present work has taken that assumption -- and discarded all events within \SI{15}{\degree} of the probe's spin axis. 

This cuts it down to ... events. 

Furthermore distinguish the fundamental mode from the second harmonic...

Insist on a certain confidence... 



% -----------------------------------------------------------------------------
% -----------------------------------------------------------------------------
% -----------------------------------------------------------------------------
\subsection{Identifying the Fundamental Harmonic}

Takahashi\cite{takahashi_2013} spells this out explicitly. Fundamental mode, south of the equator, V should lead B by 90 degrees. 

Per Takahashi\cite{takahashi_2011}, phase lag can be used to distinguish the fundamental harmonic from the second harmonic. 

Chisham and Orr\cite{chisham_1991} argue that around \SI{7}{\RE}, frequency around \SI{10}{\mHz} precludes higher harmonics. Or maybe look at \cite{green_1985}?

Green\cite{green_1979} and Cummings\cite{cummings_1969} talk about the frequencies for ideal toroidal modes. 

Dai\cite{dai_2015} says to look at \cite{takahashi_2011} and \cite{dai_2013} for unambiguous identification of the fundamental mode. 

\cite{liu_2010} talks about setting the coordinate system using the time-averaged magnetic field, and the azimuthal direction being $\hat{B_0} \times \hat{r}$. 





% -----------------------------------------------------------------------------
% -----------------------------------------------------------------------------
% -----------------------------------------------------------------------------
\subsection{Bias in MLT}

Because RBSP's spin plane faces the Sun, it can't get good electric field data at dawn or dusk. Let's plot the fraction of time at each hour of MLT that the data is unusable. 









