% =============================================================================
% =============================================================================
% =============================================================================
\section{Geomagnetic Disturbances}

% -----------------------------------------------------------------------------
% -----------------------------------------------------------------------------
% -----------------------------------------------------------------------------
\subsection{Storms}

% -----------------------------------------------------------------------------
% -----------------------------------------------------------------------------
% -----------------------------------------------------------------------------
\subsection{Substorms}

Storms! CMEs, etc. 

\todo{Definition of a substorm comes from \cite{akasofu_1964}. \cite{mcpherron_1973_substorms} added the growth phase (previously it was just expansion and recovery). }

What causes a storm. 

Storm effects: outer magnetosphere, inner magnetosphere, on the ground.

The ionospheric profiles used in this model are based on values tabulated in the Appendix B of Kelley's book\cite{kelley_1989}. They were adapted by Lysak\cite{lysak_2013} to take into account the effect of the magnetosphere's latitude-dependent density profile. 

Mean molecular mass of \SI{28}{\amu} at \SI{100}{\km}, \SI{16}{\amu} around \SI{400}{\km}, down to \SI{1}{\amu} above \SI{1400}{\km}. 

Simulations are carried out using four profiles: active day, quiet day, active night, quiet night. 

Profiles are static for the duration of a simulation. Even so-called ultra low frequency waves are still much faster than convective timescales. 

\todo{Come up with a characteristic convective timescale or two, and cite it. }

The effects of mean molecular mass on conductivity are computed per the usual definitions. 
\begin{align}
  \sp &= \displaystyle\sum_s \frac{n_s q_s^2}{m_s} \frac{\nu_s}{\nu_s^2 + \Omega_s^2} &
  \sh &= -\displaystyle\sum_s \frac{n_s q_s^2}{m_s} \frac{\Omega_s}{\nu_s^2 + \Omega_s^2} &
  \sz &= \displaystyle\sum_s \frac{n_s q_s^2}{m_s \nu_s}
\end{align}

Each profile is resolved to an altitude of about $\SI{e4}{\km}$, and include well-resolved $E$, $F_1$, and $F_2$ layers. 









