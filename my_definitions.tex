




% show todo notes in red. 
\newcommand{\todo}[1]{ \textcolor{red}{TODO: #1} }
% hide todo notes. 
%\newcommand{\todo}[1]{}


% Names with special characters. 
\newcommand{\Alfven}{Alfv\'en\xspace}
\newcommand{\Ampere}{Amp\`ere\xspace}
\newcommand{\Schrodinger}{Schr\"odinger\xspace}

% To make sure the capitalization is consistent. 
\newcommand{\ohmlaw}{Ohm's Law\xspace}
\newcommand{\amplaw}{\Ampere's Law\xspace}
\newcommand{\farlaw}{Faraday's Law\xspace}

% What should Radoski's dipole coordinates be named? \nu is overused. 
\newcommand{\radx}{\ensuremath{x}\xspace}
\newcommand{\rady}{\ensuremath{y}\xspace}
\newcommand{\radz}{\ensuremath{z}\xspace}

% What should Lysak's coordinates be named? I don't love u1 u2 u3. 
\newcommand{\lysaki}{\ensuremath{u^i}\xspace}
\newcommand{\lysakj}{\ensuremath{u^j}\xspace}
\newcommand{\lysakx}{\ensuremath{u^1}\xspace}
\newcommand{\lysaky}{\ensuremath{u^2}\xspace}
\newcommand{\lysakz}{\ensuremath{u^3}\xspace}

% Coordinate names... these should probably be italicized? 
\newcommand{\x}{\ensuremath{x}\xspace}
\newcommand{\y}{\ensuremath{y}\xspace}
\newcommand{\z}{\ensuremath{z}\xspace}
\newcommand{\X}{\ensuremath{X}\xspace}
\newcommand{\Y}{\ensuremath{Y}\xspace}
\newcommand{\Z}{\ensuremath{Z}\xspace}

% Field-aligned unit vectors. 
\newcommand{\xhat}{\ensuremath{\hat{x}}\xspace}
\newcommand{\yhat}{\ensuremath{\hat{y}}\xspace}
\newcommand{\zhat}{\ensuremath{\hat{z}}\xspace}

% Spherical unit vectors. 
\newcommand{\rhat}{\ensuremath{\hat{r}}\xspace}
\newcommand{\qhat}{\ensuremath{\hat{\theta}}\xspace}
\newcommand{\fhat}{\ensuremath{\hat{\phi}}\xspace}

% Use underlines for vectors and tensors. 
\renewcommand{\vec}[1]{\underline{#1}}
\newcommand{\tensor}[1]{\underline{\underline{#1}}}

% Differential operators. 
\newcommand{\dd}[1]{\ensuremath{ \frac{\partial}{\partial #1} }\xspace}
\newcommand{\ddt}{\dd{t}\xspace}
\newcommand{\cross}[2]{\ensuremath{ \vec{#1} \! \times \! \vec{#2} }\xspace}
\newcommand{\curl}[1]{\ensuremath{ \nabla \! \times \! \vec{#1} }\xspace}

\renewcommand{\div}[1]{\ensuremath{ \nabla \cdot \vec{#1} }\xspace}
\newcommand{\grad}[1]{\ensuremath{ \nabla #1 }\xspace}

% Properly-scaled parentheses for grouping terms or for arguments. 
\newcommand{\lr}[1]{ \left( #1 \right) }
\newcommand{\lrsmall}[1]{ \left( {\scriptstyle #1} \right) }
\renewcommand{\arg}[1]{\!\lr{#1}}
\newcommand{\argsmall}[1]{\!\lrsmall{#1}}
\newcommand{\lrb}[1]{ \left[ #1 \right] }
\newcommand{\argb}[1]{\!\lrb{#1}}

% Circled plus-minus symbol. Solving quartics requires \pm and \opm. 
\newcommand{\opm}{ \text{ \textcircled{ \ensuremath{\hskip -0.2em \pm} } }\xspace}

% Define a better looking eV by moving the V slightly left
\DeclareSIUnit\electronvolt{e\hspace{-0.08em}V}

\DeclareSIUnit\RE{R_E}
\DeclareSIUnit\S{S}


\newcommand{\dt}{\ensuremath{\delta \hspace{-0.1em} t}\xspace}
\newcommand{\dr}{\ensuremath{\delta \hspace{-0.1em} r}\xspace}

% Assignment operator for pseudocode. 
\newcommand{\assign}{\ensuremath{\leftarrow}\xspace}


% Azimuthal modenumber, typically indicated with a lowercase m. 
\newcommand{\azm}{\ensuremath{m}\xspace}

% Azimuthal modenumber, typically indicated with a lowercase m. 
\newcommand{\me}{\ensuremath{m_{e}}\xspace}

% Jacobian dererminant, typically indicated with a capital J, which we are using for current. 
\newcommand{\jac}{\ensuremath{G}\xspace}

% Dispersion tensor, typically indicated with... a capital D?
\newcommand{\dispersiontensor}{\tensor{T}\xspace}


% These things just get used a lot in the dispersion relation chapter...

% Boris-corrected speed of light. 
\newcommand{\cb}{\ensuremath{c_B}\xspace}

% Boris-corrected plasma frequency. 
\newcommand{\ob}{\ensuremath{\omega_B}\xspace}

\newcommand{\op}{\ensuremath{\omega_P}\xspace}


% Boris-corrected electric constant. 
\newcommand{\eb}{\ensuremath{\epsilon_B}\xspace}

% Alfven speed. 
\newcommand{\va}{\ensuremath{v_A}\xspace}

% Perpendicular electric constant. 
\newcommand{\ep}{\ensuremath{\epsilon_\bot}\xspace}

% Epsilon zero. 
\newcommand{\ez}{\ensuremath{\epsilon_0}\xspace}

\newcommand{\mz}{\ensuremath{\mu_0}\xspace}
\newcommand{\oomz}{\ensuremath{ \frac{1}{\mz} }\xspace}

% Conductivities. 
\newcommand{\sz}{\ensuremath{\sigma_0}\xspace}
\newcommand{\sh}{\ensuremath{\sigma_H}\xspace}
\renewcommand{\sp}{\ensuremath{\sigma_P}\xspace}

\newcommand{\spe}{\ensuremath{\frac{\sigma_P}{\ep}}\xspace}
\newcommand{\she}{\ensuremath{\frac{\sigma_H}{\ep}}\xspace}

% 3x3 matrix. 
\newcommand{\mmm}[9]{ \left[ \begin{array}{ccc}
    #1 & #2 & #3 \\
    #4 & #5 & #6 \\
    #7 & #8 & #9
  \end{array} \right] }

% 2x2 matrix. 
\newcommand{\mm}[4]{ \left[ \begin{array}{cc}
    #1 & #2 \\
    #3 & #4
  \end{array} \right] }

% 2x1 matrix, or 2-vector. 
\newcommand{\vv}[2]{ \left[ \begin{array}{c}
    #1 \\
    #2
  \end{array} \right] }


% Physics constants
\newcommand{\C}{{\mathrm{c}}}

% Add space between rows of tables
\newcommand{\spacerows}[1]{\renewcommand{\arraystretch}{#1}}







