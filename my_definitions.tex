


\newcommand{\Alfven}{Alfv\'en}

\newcommand{\Ampere}{Amp\`ere}

\newcommand{\xhat}{\ensuremath{\hat{x}}\xspace}
\newcommand{\yhat}{\ensuremath{\hat{y}}\xspace}
\newcommand{\zhat}{\ensuremath{\hat{z}}\xspace}

\DeclareSIUnit\RE{R_E}

\renewcommand{\vec}[1]{\underline{#1}}

\newcommand{\tensor}[1]{\underline{\underline{#1}}}

\newcommand{\dd}[1]{\ensuremath{ \frac{\partial}{\partial #1} }\xspace}

\newcommand{\ddt}{\dd{t}\xspace}

\newcommand{\dt}{\ensuremath{\delta \, t} \xspace}

\newcommand{\curl}[1]{\ensuremath{ \nabla \times \vec{#1} }\xspace}

\newcommand{\lr}[1]{ \left( #1 \right) }

\newcommand{\lrsmall}[1]{ \left( {\scriptstyle #1} \right) }

\newcommand{\mmm}[9]{ \left[ \begin{array}{ccc}
    #1 & #2 & #3 \\
    #4 & #5 & #6 \\
    #7 & #8 & #9
  \end{array} \right] }





% Note: Do we want to squish dt together? Do we want to always have tiny spaces
% between variables being multiplied together? 
\newcommand{\deltat}{\delta \hspace{-0.1em} t}


% Physics constants
\newcommand{\C}{{\mathrm{c}}}

% Add space between rows of tables
\newcommand{\spacerows}[1]{\renewcommand{\arraystretch}{#1}}

% Define a better looking eV by moving the V slightly left
\DeclareSIUnit\electronvolt{e\hspace{-0.08em}V}
