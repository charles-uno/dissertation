% Charles McEachern

% Fall 2015

% This is a template for writing a note to Bob in TeX. 

\documentclass{article}

% =============================================================================
% ==================================== Package List Copied from Thesis Template
% =============================================================================

\usepackage{epsfig} % Allows the inclusion of eps files
\usepackage{epic} % Enhanced picture mode
\usepackage{eepic} % Extensions for epic
\usepackage{units} % SI unit typesetting
\usepackage{url} % URL handling
\usepackage{longtable} % Tables that continue onto multiple pages
\usepackage{mathrsfs} % Support for \mathscr script
\usepackage{multirow} % Span rows in tables
\usepackage{bigstrut} % Space struts in tables up and down
\usepackage{amssymb} % AMS math symbols and helpers
\usepackage{graphicx} % Enhanced graphics support
\usepackage{setspace} % Adjust spacing in captions, single by default
\usepackage{xspace} % Automatically adjusting space after macros
\usepackage{amsmath} % \text, and other math formatting options
\usepackage{siunitx} % \num{} formatting and SI unit formatting
\usepackage{booktabs} % Enhanced tables with \toprule, etc.
\usepackage{hyperref} % Add clickable links to other parts of the document
\usepackage[noabbrev]{cleveref} % Automatically determine \cref type

\usepackage{parskip} % http://ctan.org/pkg/parskip

% Configure the siunitx package
\sisetup{
    group-separator = {,}, % Use , to separate groups of digits, like 12,345
    list-final-separator = {, and } % Always use the serial comma in \SIlist
}

% Configure the cleveref package
\newcommand{\creflastconjunction}{, and } % Always use the serial comma

\linespread{1.3}

% =============================================================================
% ================================================================= Definitions
% =============================================================================




\newcommand{\Alfven}{Alfv\'en\xspace}

\newcommand{\Ampere}{Amp\`ere\xspace}

\newcommand{\xhat}{\ensuremath{\hat{x}}\xspace}
\newcommand{\yhat}{\ensuremath{\hat{y}}\xspace}
\newcommand{\zhat}{\ensuremath{\hat{z}}\xspace}

\DeclareSIUnit\RE{R_E}

\renewcommand{\vec}[1]{\underline{#1}}

\newcommand{\tensor}[1]{\underline{\underline{#1}}}

\newcommand{\dd}[1]{\ensuremath{ \frac{\partial}{\partial #1} }\xspace}

\newcommand{\ddt}{\dd{t}\xspace}

\newcommand{\dt}{\ensuremath{\delta \hspace{-0.1em} t} \xspace}

\newcommand{\curl}[1]{\ensuremath{ \nabla \times \vec{#1} }\xspace}

\newcommand{\lr}[1]{ \left( #1 \right) }

\newcommand{\lrsmall}[1]{ \left( {\scriptstyle #1} \right) }

\renewcommand{\arg}[1]{\!\lr{#1}}

\newcommand{\argsmall}[1]{\!\lrsmall{#1}}



\newcommand{\mmm}[9]{ \left[ \begin{array}{ccc}
    #1 & #2 & #3 \\
    #4 & #5 & #6 \\
    #7 & #8 & #9
  \end{array} \right] }

\newcommand{\mm}[4]{ \left[ \begin{array}{cc}
    #1 & #2 \\
    #3 & #4
  \end{array} \right] }

\newcommand{\vv}[2]{ \left[ \begin{array}{c}
    #1 \\
    #2
  \end{array} \right] }






% Note: Do we want to squish dt together? Do we want to always have tiny spaces
% between variables being multiplied together? 
\newcommand{\deltat}{\delta \hspace{-0.1em} t}


% Physics constants
\newcommand{\C}{{\mathrm{c}}}

% Add space between rows of tables
\newcommand{\spacerows}[1]{\renewcommand{\arraystretch}{#1}}

% Define a better looking eV by moving the V slightly left
\DeclareSIUnit\electronvolt{e\hspace{-0.08em}V}


% =============================================================================
% ======================================================== Document Starts Here
% =============================================================================

\begin{document}

% -----------------------------------------------------------------------------
% -----------------------------------------------------------------------------
% -----------------------------------------------------------------------------
\section{Model Summary}

We start with \farlaw, \amplaw (plus a driving term), and \ohmlaw (including electron inertial effects in the parallel direction). 
\begin{align*}
  \ddt \vec{B} & = - \curl{E}
  &
  \tensor{\epsilon} \cdot \ddt \vec{E} & = 
    \frac{1}{\mu_0} \lr{ \curl{B} } - \vec{J}_{drive} - \vec{J}
  \\
  \vec{J}_\bot & = \tensor{\sigma}_\bot \cdot \vec{E}_\bot
  &
  \ddt J_\parallel & = \frac{n e^2}{\me} E_\parallel - \nu J_\parallel
\end{align*}

We use a two-dimensional dipole grid spanning $2 \lesssim L \lesssim 10$. Field lines end at a fixed altitude, taken to be the ionospheric current sheet, at $\sim \SI{100}{\km}$. 





At the boundary, we assume an insulating atmosphere and a conducting Earth. This allows us to compute a scalar magnetic potential, $\Psi$. Each time step, we use $B_r$ to compute $\Psi$ in terms of spherical harmonics. We then use $\Psi$ to compute $B_\phi$ and $B_\theta$ at $R_E$ and $R_I$. From the jump in magnetic field over the ionospheric current sheet, we compute boundary values for the horizontal electric field. 
\begin{align*}
  \mu_0 \tensor{\Sigma} \cdot \vec{E} & = \hat{r} \times \Delta \! \vec{B} 
\end{align*}

We use static profiles for conductivity, \Alfven speed, number density, and so on. We have several profiles, allowing us to simulate the dayside or nightside during quiet or active times. 


% -----------------------------------------------------------------------------
% -----------------------------------------------------------------------------
% -----------------------------------------------------------------------------
\section{Model Strengths}

This 











%The parallel electric field, and all components of the magnetic field, are updated directly. Perpendicular currents are eliminated. Perpendicular electric fields and the parallel current are solved with integrating factors. 
%\begin{align*}
%  \vec{E}_\bot \! \arg{\dt} & = 
%    \tensor{R} \argsmall{\frac{\sigma_H}{\epsilon_\bot} \dt} \cdot \vec{E}_\bot \! \arg{0}
%      \exp \argsmall{-\frac{\sigma_P}{\epsilon_\bot} \dt}
%    + \dt \tensor{R} \argsmall{\frac{\sigma_H}{\epsilon_\bot} \frac{\dt}{2} } \cdot  \vec{F}_\bot \! \argsmall{ \frac{\dt}{2} } \exp \argsmall{-\frac{\sigma_P}{\epsilon_\bot} \frac{\dt}{2} }
%  \\
%  J_\parallel \arg{\dt} & = 
%    J_\parallel \arg{0} \exp \arg{ -\nu \dt }
%    + \frac{n e^2}{\me} \dt \, E_\parallel \argsmall{\frac{\dt}{2}}  
%    \exp \argsmall{ -\nu \frac{\dt}{2} }
%\end{align*}

%where
%\begin{align*}
%  \vec{F} \equiv \frac{1}{\mu_0 \epsilon_\bot} \curl{B} - \frac{1}{\epsilon_\bot} \vec{J}_{drive}
%  \qquad \text{and} \qquad
%  \tensor{R} \arg{\theta} \equiv \mm{\cos\theta}{-\sin\theta}{\sin\theta}{\cos\theta}
%\end{align*}



\end{document}













