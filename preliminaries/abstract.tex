
Field line resonances --- that is, \Alfven waves bouncing between the northern
and southern foot points of a geomagnetic field line --- serve to energize
magnetospheric particles through drift-resonant interactions, carry energy from
high to low altitude, induce currents in the magnetosphere, and accelerate
particles into the atmosphere. Wave structure and polarization significantly
impact the execution these roles. The present work showcases a new two and a
half dimensional code, Tuna, ideally suited to model FLRs, with the ability to
consider large-but-finite azimuthal modenumbers, coupling between the poloidal,
toroidal, and compressional modes, and arbitrary harmonic structure. Using
Tuna, the interplay between Joule dissipation and poloidal-to-toroidal rotation
is considered for both dayside and nightside conditions. An attempt is also
made to demystify giant pulsations, a class of FLR knows for its distinctive
ground signatures. Numerical results are supplemented by a survey of \about500
FLRs using data from the Van Allen Probes, the first such survey to
characterize each event by both polarization and harmonic. The combination of
numerical and observational results suggests an explanation for the disparate
distributions observed in poloidal and toroidal FLR events. 



%Something something Pc4 pulsations. Ultra low frequency (ULF) waves with periods of a minute or two. They correspond to field line resonances near Earth's plasmapause. Drift or drift-bounce resonance with energetic radiation belt and ring current particles. Radial diffusion. 
%Pc4 pulsations are known to exhibit different behavior based on their azimuthal modenumber. Low-\azm Pc4 pulsations are driven at the outer edge of the magnetosphere, and have a compressional component. High-\azm Pc4 pulsations are non-compressional and are driven within the magnetosphere. 
%Giant pulsations are a subset of high-\azm Pc4 pulsations which are of particular interest. 
%High-\azm Pc4 pulsations are hard to simulate. Traditionally, simulations are driven from the outer boundary, which doesn't work in this case. And resolving high azimuthal modenumbers is computationally expensive in a 3D simulation. 
%\cref{ch_intro} gives a general introduction is made to Earth's magnetosphere. Also summarizes Pc4 pulsations in terms of prominent theoretical and observational work, including work on giant pulsations. (Note: this mostly doesn't exist yet.) 
%\cref{ch_model} presents a 2.5D model designed to simulate Pc4 pulsations, including those with high azimuthal modenumbers. The model is built upon recent work by Lysak, and includes a dipole-aligned geometry, a height-resolved ionosphere, and coupling to a conducting Earth. Ring current modulation is introduced as a novel driving mechanism. Also investigates the results of adding electron inertial effects to the model, allowing the computation of parallel electric fields and field-aligned currents. It also derives dispersion relations for a cold, resistive plasma, such as Earth's inner magnetosphere and ionosphere. This gives an idea as to the waves expected to be produced. 
%\cref{ch_results} showcases changes in Pc4 behavior as a result of altering the azimuthal modenumber. This includes a rotation of poloidal to toroidal waves, significantly refining past work by Mann, and before that Radoski. (There are a lot of figures here. It might end up being two chapters.)
%\cref{ch_rbsp} compares model output to poloidal Pc4 observations made with the Van Allen Probes. (Note: this hasn't quite happened yet. The chapter briefly explains.)
%\cref{ch_conclusion} offers a summary and possibilities for future work. 
%NOTE: ALL FIGURES ARE PDFS. Some of the frames are small, but they should remain sharp when you zoom in on them. 
